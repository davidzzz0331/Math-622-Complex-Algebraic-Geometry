\documentclass{article}
\usepackage[utf8]{inputenc}
\usepackage{amsmath}
\usepackage{amsfonts}
\usepackage{amssymb}
\usepackage{tikz}
\usepackage{fullpage}
\usepackage{tikz-cd}
\usepackage{spectralsequences}
\usepackage{adjustbox}
\usepackage{xfrac}
\usepackage{tcolorbox}
\usepackage{xcolor}
\usepackage{hyperref}
\usepackage{graphicx}
\graphicspath{ {D:/Chrome Downloads./} }
\usepackage[parfill]{parskip}
\usepackage{amsthm}
\usetikzlibrary{calc}
\theoremstyle{definition}
\newtheorem{theorem}{Theorem}[section]
\theoremstyle{definition}
\newtheorem{definition}{Definition}[theorem]
\theoremstyle{definition}
\newtheorem{remark}{Remark}[theorem]
\theoremstyle{definition}
\newtheorem{proposition}{Proposition}[theorem]
\theoremstyle{definition}
\newtheorem{lemma}[theorem]{Lemma}
\theoremstyle{definition}
\newtheorem{corollary}{Corollary}[theorem]
\theoremstyle{definition}
\newtheorem{example}{Example}[theorem]
\newtheorem{problem}{Problem}





\tikzset{curve/.style={settings={#1},to path={(\tikztostart)
    .. controls ($(\tikztostart)!\pv{pos}!(\tikztotarget)!\pv{height}!270:(\tikztotarget)$)
    and ($(\tikztostart)!1-\pv{pos}!(\tikztotarget)!\pv{height}!270:(\tikztotarget)$)
    .. (\tikztotarget)\tikztonodes}},
    settings/.code={\tikzset{quiver/.cd,#1}
        \def\pv##1{\pgfkeysvalueof{/tikz/quiver/##1}}},
    quiver/.cd,pos/.initial=0.35,height/.initial=0}
\title{Complex AG HW}
\author{David Zhu}

\begin{document}
\maketitle

\tableofcontents


\section{Homework 1}
\begin{problem}
    Show $\mathbb{P}^1\cong S^2$ and that $\mathbb{P}^1$ is simply connected.
\end{problem}

\begin{proof}
    Both are one-point compactification of $\mathbb{C}$: for $\mathbb{P}^1$ we identify the standard affine open $U_1:=\{[1:c]: c\in \mathbb{C}\}$ with $\mathbb{C}$ and $[0:1]$ as the point at infinity; for $S^2$ it is the standard stereographical projection argument. By universal property of one-point compactification, $\mathbb{P}^1$ and $S^2$ are homeomorphic. Thus, $\mathbb{P}^1$ is simply connected since $S^2$ is simply connected.   
\end{proof}

\begin{problem}
    Compute the algebraic dimension of $\mathbb{P}^2$ and of $\mathbb{C}/(\mathbb{Z}+i \mathbb{Z})$
\end{problem}
\begin{proof}
    For $\mathbb{P}^2$: we claim the algebraic dimension for $\mathbb{P}^2$ is $2$. From Siegel's theorem, it suffices to show $a(\mathbb{P}^2)\geq 2$. Define 
    \[f_i: \mathbb{P}^2\to \coprod_{x\in \mathbb{P}^2}Q(\mathcal{O}_{\mathbb{P}^{2},x} )\]
    by sending a point $p$ to the germ of the rational function $\frac{z_i}{z_0}\in Q(\mathcal{O}_{\mathbb{P}^2,p})$ for $i=1,2$. Locally on $U_0\cong \mathbb{A}^2=\{ (u,v):=(\frac{z_1}{z_0}, \frac{z_2}{z_0})\}$, the assignment $f_i$ are just the coordinate functions; on $U_1\cong \mathbb{A}^2=\{ (u,v):=(\frac{z_0}{z_1}, \frac{z_2}{z_1})\}$, we have $f_1=\frac{1}{u}$ and $f_2= \frac{v}{u}$. The change of coordinates with $U_2$ is similar. Clearly, $\frac{z_1}{z_0}$ and $\frac{z_2}{z_0}$ do not satisfy any algebraic relations by degree reasons, so the function field is of trascendence degree at least $2$. 


    For the complex torus $\mathbb{C}/\mathbb{Z}+i \mathbb{Z}$, we know the field of meromorphic functions on the torus is generated by the Weierstrass elliptic functions and its derivative, which satisfy some algebraic relations. Therefore the function field is of trascendence degree $1$, and 
    \[a(T)=1\]

\end{proof}


\begin{problem}
    Prove that holomorphic maps $\mathbb{P}^1 \to$  complex tori are constant.
\end{problem}

\begin{proof}
    Since $\mathbb{P}^1$ is simply-connected, we have a lift
    \[\begin{tikzcd}
        & {\mathbb{C}} \\
        {\mathbb{P}^1} & {\mathbb{T}^1}
        \arrow[from=1-2, to=2-2]
        \arrow["{\tilde{f}}", dashed, from=2-1, to=1-2]
        \arrow["f", from=2-1, to=2-2]
    \end{tikzcd}\]
The holomorphicity of $\tilde{f}$ is inherited from that of $f$; however, since $\mathbb{P}^1$ is compact and connected, such $\tilde{f}$ must be a constant. For higher dimensional tori, the argument is the same but replacing $\mathbb{C}$ with $\mathbb{C}^n$, noting that on each coordinate $\tilde{f}$ still restricts to a constant function. 

\end{proof}








\begin{problem}
    For the Hopf manifold in $n=1$, identify it as an elliptic curve and determine its lattice.
\end{problem}
\begin{proof}
    Suppose Hopf manifold is given by the $\mathbb{Z}$-action $z\mapsto \lambda^k z$. Consider the exponential 
    \[f: \mathbb{C}\to \mathbb{C}^*\]
    given by by $z\mapsto e^{2\pi i z}$ which is holomorphic. Note that $f$ induces a holomorphic map
    \[f': \mathbb{C}/(\mathbb{Z}+\frac{\textrm{log}(\lambda)}{2\pi i} \mathbb{Z}) \to \mathbb{C}^*/\mathbb{Z} \]
It is easy to check that $f'$ is well-defined with respect to the lattices. Thus, it suffices to show $f'$ is bijective. The surjectivity is easy since $f$ is surjective; for injectivity, suppose we have $z_1=x_1+iy_1$ and $z_2=x_2+iy_2$ such that 
\[e^{2\pi i z_1} = \lambda ^k e^{2\pi i z_2}\]
Comparing real parts yields 
\[\lambda^k e^{-2 \pi y_2}=e^{-2\pi y_1}\]
which implies $y_1-y_2=\frac{k \textrm{log}(\lambda)}{2\pi}$; the imaginary part yields
\[e^{2\pi i x_1}=e^{2\pi_i i x_2}\]
which implies $x_1-x_2=n$ for some $n\in \mathbb{Z}$. Thus, $z_1$ and $z_2$ must be in the same coset in $\mathbb{C}/(\mathbb{Z}+\frac{\textrm{log}(\lambda)}{2\pi i} \mathbb{Z})$.



\end{proof}







\begin{problem}
    Explain how to recover the bundle $E$ corresponding to a locally free sheaf $F$. Define the sheaf
    $\mathcal{O}(k)$ on projective space and check that it corresponds to the correct bundle.
    
\end{problem}
\begin{proof}
    Let $\mathcal{F}$ be a locally free sheaf on $X$. For every point $x\in X$, choose a neighborhood $U_x$ where $\mathcal{F}|_{U_x}$ is isomorphic to $\mathcal{O}_{U_x}^{n}$. Let each $U_x$ be a trivializing neighborhood for a rank $n$ bundle, and the sheaf transition maps 
    \[\mathcal{O}_{U_x\cap U_y}^n\to \mathcal{O}_{U_x\cap U_y}^n\]
    is given by an $n\times n$ matrix of holomorphic functions, which gives us the bundle structure maps. These provide the data for constructing a holomorphic vector bundle over $X$. 

    The locally free sheaf of rank $1$ $\mathcal{O}(k)$ on $\mathbb{P}^n$ can be constructed as follows: on each affine open $U_i$, we define $\mathcal{O}(k)|_{U_i}$ to be the sheaf of holomorphic functions on $U_i$. The transition functions 
    \[\mathcal{O}(k)|_{U_i\cap U_j}\to \mathcal{O}(k)|_{U_i\cap U_j}\]
    are given by multiplying by $(\frac{z_i}{z_j})^m$. 






\end{proof}






\begin{problem}
    Show that the Grassmannian $\textrm{Gr}(2,4)$ is a $4$-dimensional quadric. (Use the Plücker embedding)
\[\textrm{Gr}(2,4)\to \mathbb{P}^5\]
\end{problem}

\begin{proof}
    This is just the Plucker embedding
    \[ \textrm{Gr}(2,4)\to \mathbb{P}(\bigwedge^2 \mathbb{C}^4)\cong \mathbb{P}(\mathbb{C}^6)=\mathbb{P}^5\]
    and the image satisfies the single Plucker equation: 
    \[\textrm{Gr}(2,4)=V(p_{12}p_{34}-p_{13}p_{24}+p_{14}p_{23})\subset \mathbb{P}^5\]
    By dimension reasons, $\textrm{Gr}(2,4)$ must be the hypersurface cut out by this single equation, hence a quadric.
\end{proof}






\begin{problem}
    Explain why the flag varieties are projective varieties.
\end{problem}

\begin{proof}
    They embed in a product of Grassmannian, which via Plucker embedding embeds into a product of projective spaces, which embeds into a projective space of larger dimension via the Segre embedding. 
    \[\textrm{Fl}(V,k_1,...,k_n)\hookrightarrow \textrm{Gr}(k_1,V)\times ...\textrm{Gr}(k_n,V)\xrightarrow{\textrm{Plucker}}\mathbb{P}(\wedge^{k_1}V)\times...\times \mathbb{P}(\wedge^{k_n}V)\xrightarrow{\textrm{Segre}} \mathbb{P}^N \]

The image of the first map are the points $(V_1,V_2,...,V_n)$ where the vector spaces satisfies $V_i\subset V_{i+1}$. Representing these vector space by matrices $A_i$, the inclusion criteria corresponds to the existence of a $k_i\times k_{i+1}$ matrix $X$ such that $A_{i}X=A_j$. This relation is now easily seen to be algebraic, defined by a set of polynomials, thus holomorphic. 



\end{proof}


\section{Homework 2}

\begin{problem}
    Let $P:=\mathbb{P}^n$, $L:=O_P(d)$. Show that the map $\varphi_L$ is a morphism iff $d\geq 0$, an isomorphism iff $d=1$, and an embedding iff $d\geq 1$. the image is called the Veronese variety. For a hyperplane
    $H\subset P$, what is $\varphi ^{-1}(H)$?
\end{problem}
\begin{proof}
    Recall that the dimension of $H^0(P^n; O(d))$ corresponds to the dimension of $\mathbb{C}[x_0,....,x_n]_d$ as a $\mathbb{C}$-vector space. In particular, it is of dimension $N:=\binom{n+d}{d}$ if $d\geq 0$. The morphism $\varphi_L$ is thus defined via choosing a basis of monomials
    $\{s_0,...,s_{N-1}\}$, and define 
    \[\varphi_L: \mathbb{P}^n\to \mathbb{P}^N\]
    by 
    \[x\mapsto [s_0(x):...: s_{N-1}(x)]\]
    locally each $s_i$ is a polynomial, so this indeed defines a morphism; for $d=0$ this is a constant map into $\mathbb{C}$; for $d=1$ we have a linear automorphism of $\mathbb{P}^n$; for $d\geq 2$, one sees that if 
    \[[x_0^d: x_1x^{d-1}_0:...:x_{n}^d]=[y_0^d: y_1y^{d-1}_0:...:y_{n}^d]\]
    then $x_i$ and $y_i$ differ by the same $d$th root of unity, so we have injectivity.

    It is immediate from definition that the preimage of a degree $d$ hypersurface in $\mathbb{P}^n$ is a hyperplane in $\mathbb{P}^n$; similarly, the image of a hyperplane in $\mathbb{P}^n$ is a degree $d$ hypersurface under the Veronese embedding. 








\end{proof}











\begin{problem}
In the above setting, let $p\in P$ and let $V\subset H^0(P, O(d))$ be the linear system of sections
of $O_p(d)$ that vanish at $p$. Show that $\varphi_V$ is no longer a morphism on $P$ but only on $P\setminus p$.
Describe the closure of its image.
\end{problem}
\begin{proof}
    The map $\varphi_L$ is defined the same way as in the previous problem on $\mathbb{P}^n\setminus p$, and since everything is checked locally, $\varphi_L$ is still well-defined and holomorphic. The closure of the image can be constructed via blowing up $\mathbb{P}^n$ at $p$, denoted by $\tilde{P}:=\textrm{Bl}_p \mathbb{P}^n$. Then, the section of $V$ pullback to a base-point free linear system on $\tilde{P}$, which induces a morphism 
    \[\tilde{\varphi}: \tilde{P}\to \mathbb{P}^{N'}\]
    The closure of the image of $\varphi_L$ is exactly the image of $\tilde{\varphi}$. 
\end{proof}



















\begin{problem}
    Let $P:= \mathbb{P}^2$ with homogeneous coordinates $x,y,z$ let $p\in P$ be the point $[0:0:1]$, and let
$C\subset P$ be the curve with equation $x_1^2x_2^2=x_0^4$. Let $Y$ be the blowup of $P$ at $p$. Describe the
strict and total transforms in $Y$ of $C$. What happens if you blow up a second time?
\end{problem}
\begin{proof}
    Let $\tilde{P}$ denote the blowup of $\mathbb{P}^2$ at $[0:0:1]$. Equation-wise, $\tilde{P}$ is the variety 
    \[\{ ([x_0:x_1:x_2],[z_0:z_1])\in \mathbb{P}^2\times \mathbb{P}^1:x_0z_1=x_1z_0 \}\]
and the structure morphism 
\[\pi: \tilde{P}\to P\]
is the projection onto the first factor. 

The total transform by definition is 
\[\{ ([x_0:x_1:x_2],[z_0:z_1])\in \mathbb{P}^2\times \mathbb{P}^1:x_0z_1=x_1z_0, x_0^4=x_1^2x_2^2 \}\]
The strict transform of $C$: on the chart $z_0\neq 0$, we subsitute $x_0\frac{z_1}{z_0}=x_1$ into $x_1^2x_2^2=x_0^4$, which gives us $z_1^2x_2^2=x_0^2z_0^2$ after dividing both sides by $x^2_0$ (which can be seen as throwing away the $+2E$ part coming from the exceptional divisor). Thus, the strict transform is 

\[\{ ([x_0:x_1:x_2],[z_0:z_1])\in \mathbb{P}^2\times \mathbb{P}^1:x_0z_1=x_1z_0, z_1^2x_2^2=x_0^2z_0^2\}\]
We get two components meeting transversely at a single point $$([0:0:1],[1:0])$$ with multiplicity $2$ on the exceptional divisor.

To prepare for the second blowup, it is easier to look at the local piture: in the affine open where $x_2=1$ and $z_0=1$, the strict transform of C after the first blowup is defined by $x_0z_1=x_1$ and $z_1^2=x_0^2$. Blowing up at the origin, this the two component of the strict transform now separate and meet at different points on the exceptional divisor and resolve the singularity. 



\end{proof}





















\begin{problem}


    Let $(X,\omega)$ be compact connected Kähler, $\textrm{dim}X\geq 2$, and $h$ its Kähler metric. Show: wedge with $\omega$ is injective on 
$1$-forms. Deduce: if $\varphi$ is a a smooth function and $\varphi h$ is also a Kähler metric, then $\varphi$ is constant.
\end{problem}
\begin{proof}




    Let $\omega$ be a symplectic form on a manifold of dimension $2n \geq 4$. By Darboux's theorem, locally $\omega$ can be expressed in the normal form 
    \[\omega=dx_1\wedge dy_1+...+dx_n\wedge dy_n\]
    An arbitrary $1$-form is of the form
    \[\gamma=a_1dx_1+b_1dy_1+...b_ndy_n\]

The wedge is then $\omega\wedge \gamma=\sum^n_{i=1}\sum_{i\neq j}a_idx_i\wedge dx_j\wedge dy_j+ b_idy_i\wedge dx_j\wedge dy_j$ which vanishes iff all $a_i$ and $b_i$ vanish. 

Suppose $\varphi\circ h$ is a Kähler metric, which means 
\[d(\textrm{Im}(\varphi h))=d(\varphi \textrm{Im} h)=0\]
By Leibniz rule, we have 
\[d\varphi\wedge \textrm{Im} h+\varphi \wedge d(\omega)=0\]
which implies $d\varphi =0$ by our injectivity result. Thus, $\varphi$ is constant. 




\end{proof}







\begin{problem}
    For a holomorphic vector bundle $E$ of rank $r$ and a line bundle $L$, prove 
    \[P(E^*\otimes L^*)=P(E^*)\]
    and 
    \[O_{P(E^*\otimes L^*)}(1)\cong O_{P(E^*)}(1)\otimes \pi^*L\]
\end{problem}
\begin{proof}
For notational simplicity, we get rid of the dual everywhere except replacing $\pi^*L$ with $\pi^*L^*$. A projective bundle is determined by trivialization $\varphi_i: U_i \to P(V)$ and transition functions 
\[g_{ij}: U_i\cap U_j \to \textrm{Aut}(\mathbb{P}(V))=\textrm{PGL}(V)\]
So take a common trivilization of $E$ and $L$, and let $\{f_{ij}\}$, $\{g_{ij}\}$ be the respective transition functions. The transition function for $E\otimes L$ is then 
\[f_{ij}\cdot g_{ij}\in \textrm{GL}(V)\]
and projectivization simply passes the class $f_{ij}\cdot g_{ij}\in \textrm{GL}(V)$ to $\textrm{PGL}(V)$, where we immediately see the constant $g_{ij}$ does not matter, i.e $P(E\otimes L)$ is isomorphic to $P(E)$. Explicitly, a line in $E\otimes L$ spanned by $[e\otimes v]$ is send to the line spanned by $[e]$. 


An isomorphism of tautological bundles 
\[O_{P(E\otimes L)}(-1)\cong O_{P(E)}(-1)\otimes \pi^*L\]
can also be deduced by comparing transition functions: tensoring $L$ precisely changes the transition function by a the constant factor $g_{ij}.$ Taking dual of both sides gives us the second part of the problem. 









\end{proof}








\begin{problem}
The space of conics in the plane $\mathbb{P}^2$ is parametrized by a $\mathbb{P}^5$. This has a stratification into three strata. If we interpret $\mathbb{P}^5$ as $\mathbb{P}(\textrm{Sym}^2 V^*)$ where $\mathbb{P}^2=\mathbb{P}(V)$, then the strata are distinguished by the rank of the symmetric $3\times 3$ matrix. If we think of $\mathbb{P}^5$ as parametrizing conics, the $3$ strata are characterized by the topology of the conic: double lines, line pairs, and non-singular conics, respectively.

How far can you push the analogous stratification of the space of cubics in the plane? Focusing on the topology of the cubic, you should get a finite number of strata. One of these is Zariski-open (i.e. open and dense). Try to prove that, for each non-open stratum S, all the cubics corresponding to points of S are isomorphic to each other. (You’ll need to agree on what ‘isomorphic’ means here.) Show also that this is false for the open stratum: all the corresponding cubics are topologically the same (they are tori), but they are not all isomorphic to each other.

Hint: you may want to consider the action of  $\textrm{SL}(V) = \textrm{SL}(3,\mathbb{C})$ on everything.


\end{problem}

\begin{proof}
    First, we classify all the reducible cubics: there are only finitely many ways to factor degree $3$ homogeneous polynomials 
    \begin{enumerate}
        \item a triple line (e.g $x^3$)
        \item a double line plus a simple line (e.g $x^2y$)
        \item a line plus an irreducible conic (e.g $x(xy+z^2)$, $z(xy+z^2)$)
        \item three lines (e.g $xyz$, $xy(x+y)$)
    \end{enumerate}
Moreover, every conic in each category is isomorphic to one of the example given via an action of $SL(3, \mathbb{C})$. However, if we do not restrict to linear isomorphisms, the two classes in $3$ and $4$ are isomorphic as varieties.

For the irreducible cubics, the classical theory allows one to put them in three categories 
\begin{enumerate}
    \item Cuspidal 
    \item Nodal
    \item Elliptic curve
\end{enumerate}
where the first two are singular and the third is smooth.

The open-dense stratum is the smooth cubics, since it is the complement of singular curves, which is the zero set of a set of polynomials. Topologically, the elliptic curves are all tori since they are of genus $1$. The classical theory of elliptic curves completely classifies their isomorphism classes via the j-invariant. However, up to linear isomorphisms, there are many more isomorphisms classes. The degeneration relation of the strata looks like

\[\textrm{Smooth cubics} \to \textrm{Nodal or Cuspidal }\to \textrm{Conic + Line} \to \textrm{Three Lines}\to \textrm{ double line simple line}\to \textrm{triple line}\]






\end{proof}

\section{Homework 3}

\begin{problem}
    Let $X$ be a compact complex curve.  
Let $\mu$ be a volume form on $X$. We can consider $\mu$ as a closed form of type $(1,1)$ on $X$.

\subsection*{(a)}
By considering the integral $\int_X \mu$, show that $\mu$ is not $\overline{\partial}$-exact.  
Deduce from this that $H^1(X, K_X) \neq \{0\}$ and admits a surjective map
\[
\operatorname{Tr} : H^1(X, K_X) \longrightarrow \mathbb{C}, \quad 
\omega \longmapsto \int_X \omega.
\]

Let $D = \sum_i x_i$ be a divisor of $X$ (all of whose multiplicities are equal to $1$).  
We denote by $K_X(D)$ the holomorphic line bundle (or sheaf of free $\mathcal{O}_X$-modules of rank $1$), whose sections are:
- the holomorphic forms of degree $1$ on any open set not containing any of the $x_i$’s, and  
- in the neighbourhood of each $x_i$, the meromorphic forms which can be written as
\[
    \varphi(z_i)\frac{ dz_i}{z_i},
\]
where $z_i$ is a local coordinate centred at $x_i$ and $\varphi$ is holomorphic.

\begin{proof}
    Suppose $\mu = \overline{\partial} \alpha$ for some $(1,0)$-form $\alpha$. Then by Stokes' theorem,
\[
\int_X \mu = \int_X \overline{\partial} \alpha = 0,
\]
contradicting that $\mu$ is a volume form. Hence $\mu$ is not $\overline{\partial}$-exact.

Since $\mu$ is a closed $(1,1)$-form which is not $\overline{\partial}$-exact, it defines a nonzero cohomology class in
\[
H^{1,1}_{\overline{\partial}}(X) \cong H^1(X, K_X).
\]
Therefore,
\[
H^1(X, K_X) \neq \{0\}.
\]

Moreover, the map
\[
\operatorname{Tr}: H^1(X, K_X) \longrightarrow \mathbb{C}, \quad \omega \longmapsto \int_X \omega
\]
is well-defined and surjective, because $\mu$ itself represents a nonzero class.
\end{proof}



\subsection*{(b)}
Show that we have an exact sequence
\[
0 \longrightarrow K_X \longrightarrow K_X(D) 
\xrightarrow{\ \operatorname{Res}\ } \bigoplus_i \mathbb{C}_{x_i} \longrightarrow 0, \tag{4.15}
\]
where each sheaf $\mathbb{C}_{x_i}$ is a \emph{skyscraper sheaf} supported at $x_i$, whose group of sections over any open set not containing $x_i$ is $\{0\}$ and whose group of sections on an open set containing $x_i$ is $\mathbb{C}$.

Here the map $\operatorname{Res}_i : K_X(D) \to \mathbb{C}_{x_i}$ sends a meromorphic form $\omega$ to its residue at $x_i$, defined as
\[
\operatorname{Res}_i(\omega) = \frac{1}{2 i \pi} \int_{\partial D_i} \omega,
\]
where $D_i$ is a disk centred at $x_i$ not containing any of the $x_j$’s, $i \neq j$.
\begin{proof}

\medskip
\textbf{Exactness:}
\begin{itemize}
    \item Injectivity of $K_X \hookrightarrow K_X(D)$ is clear: a holomorphic 1-form is also a meromorphic 1-form with no poles.
    \item Exactness at $K_X(D)$: the kernel of $\operatorname{Res}$ consists of meromorphic forms whose residues vanish at all $x_i$, i.e., holomorphic everywhere, hence exactly $K_X$.
    \item Surjectivity of $\operatorname{Res}$: any tuple $(c_i) \in \bigoplus_i \mathbb{C}_{x_i}$ can be realized as residues of a meromorphic form with simple poles at $x_i$, so the map is surjective.
\end{itemize}
\end{proof}





\subsection*{(c)}
Let 
\[
\delta : \bigoplus_i \mathbb{C} = H^0\big(X, \bigoplus_i \mathbb{C}_{x_i}\big) 
\longrightarrow H^1(X, K_X)
\]
be the connecting homomorphism appearing in the long exact sequence associated to the short exact sequence \((4.15)\).

Show that $\delta(1_{x_i})$ is the class in $H^1(X, K_X)$ of the form $\bar{\partial}\mu_i$, where $\mu_i$ is a differential form of type $(1,0)$, which is $C^\infty$ away from $x_i$, and equal to $\dfrac{dz_i}{z_i}$ in a neighbourhood of $x_i$.
\begin{proof}
    Consider the short exact sequence
\[
0 \longrightarrow K_X \longrightarrow K_X(D) 
\xrightarrow{\ \operatorname{Res}\ } \bigoplus_i \mathbb{C}_{x_i} \longrightarrow 0,
\]
which gives the long exact sequence in cohomology
\[
0 \longrightarrow H^0(X,K_X) \longrightarrow H^0(X,K_X(D)) 
\xrightarrow{\operatorname{Res}} \bigoplus_i \mathbb{C} \xrightarrow{\delta} H^1(X,K_X) \longrightarrow \cdots
\]

Let $1_{x_i} \in \bigoplus_i \mathbb{C}$ be the section that is $1$ at $x_i$ and $0$ elsewhere. Choose a small disk $D_i$ around $x_i$ with local coordinate $z_i$, and let
\[
\mu_i = \frac{dz_i}{z_i}
\]
be a meromorphic 1-form representing the residue $1$ at $x_i$. Extending $\mu_i$ to a globally defined $(1,0)$-form on $X$ that is smooth away from $x_i$ (e.g., using a smooth cutoff function), we obtain a form $\tilde{\mu}_i$ such that
\[
\delta(1_{x_i}) = [\bar{\partial} \tilde{\mu}_i] \in H^1(X,K_X).
\]
Here $\tilde{\mu}_i$ equals $\frac{dz_i}{z_i}$ near $x_i$ and is $C^\infty$ elsewhere, as required.
\end{proof}




\subsection*{(d)}
Show that 
\[
\int_X \delta\mu_i = - 2 i \pi.
\]
Deduce from the long exact sequence associated to the short exact sequence \((4.15)\) the following result:

\medskip
\noindent
If $\omega$ is a meromorphic $1$-form on $X$ having poles of order at most $1$ at each $x_i$, and holomorphic otherwise, then
\[
\sum_i \operatorname{Res}_i(\omega) = 0.
\]
\end{problem}
\begin{proof}
    Let $\delta(1_{x_i}) = [\bar{\partial} \mu_i] \in H^1(X,K_X)$ as in the previous problem, where $\mu_i$ is smooth away from $x_i$ and equals $\frac{dz_i}{z_i}$ near $x_i$.  

By Stokes' theorem, choosing a small disk $D_i$ around $x_i$,
\[
\int_X \delta(1_{x_i}) = \int_X \bar{\partial} \mu_i 
= \int_{\partial D_i} \mu_i = \int_{\partial D_i} \frac{dz_i}{z_i} = 2 \pi i.
\]

Taking orientation into account (for the standard orientation of $X$), we get
\[
\int_X \delta(1_{x_i}) = - 2 \pi i.
\]

\medskip
Now consider the long exact sequence associated to 
\[
0 \longrightarrow K_X \longrightarrow K_X(D) \xrightarrow{\operatorname{Res}} \bigoplus_i \mathbb{C}_{x_i} \longrightarrow 0.
\]
The sequence in cohomology reads
\[
0 \longrightarrow H^0(X,K_X) \longrightarrow H^0(X,K_X(D)) 
\xrightarrow{\operatorname{Res}} \bigoplus_i \mathbb{C} \xrightarrow{\delta} H^1(X,K_X) \longrightarrow \cdots
\]

If $\omega \in H^0(X,K_X(D))$ is a meromorphic 1-form with simple poles at $x_i$ and holomorphic elsewhere, then
\[
\operatorname{Res}(\omega) = (\operatorname{Res}_i(\omega))_i \in \bigoplus_i \mathbb{C}.
\]

Since $\omega$ is a global section of $K_X(D)$, it lies in the kernel of $\delta$, hence
\[
\delta\big( (\operatorname{Res}_i(\omega))_i \big) = 0 \in H^1(X,K_X).
\]

Applying the trace map $H^1(X,K_X) \to \mathbb{C}$ given by integration, we obtain
\[
\sum_i \operatorname{Res}_i(\omega) = 0.
\]

\end{proof}








\begin{problem}
    For a complex manifold $C$ and an integer $d\geq 2$, define the symmetric product $C^{(d)}$ to be the quotient
of the Cartesian product $C^d$ by the action of the symmetric group $S_d$ permuting the factors. Show
that $C^{(d)}$ is smooth iff $\textrm{dim}(C)\leq 1$.
\end{problem}
\begin{proof}
    If $\textrm{dim}(C)=1$, then locally $C$ is diffeomorphic to a product of complex planes $\mathbb{C}^1\times ...\times \mathbb{C}^1$. We define a chart 
    \[\varphi: \mathbb{C}^1\times ...\times \mathbb{C}^1/S_d\to \mathbb{C}^d\]
    by sending an unordered $d$-tuple $\{x_1,...,x_d\}$ to the ordered coefficients of the monic polynomial $f(z)=z^d+a_1z^{d-1}+...+a_d$ with $\{x_1,...,x_d\}$ as presribed zeroes. Each coordinate function is a elementary symmetric polynomial, so $\varphi$ is smooth. It is clearly a bijection by construction. 

For $\textrm{dim}=n \geq 2$, consider the point of \( C^{(d)} \) corresponding to the \( d \)-tuple
    \[
    (p, p, p_3, \ldots, p_d),
    \]
    where the points \( p_i \) are distinct and \( p \) is repeated twice.  
    The stabilizer of this point in \( S_d \) contains the transposition \( (12) \). Choose local coordinates near \( p \),
    \[
    \phi : U \to \mathbb{C}^n, \quad \phi(p) = 0.
    \]
    Then a neighborhood of our point in \( C^d \) looks like
    \[
    (\mathbb{C}^n)^d = (\mathbb{C}^n_x \times \mathbb{C}^n_y) \times (\text{other factors}),
    \]
    and the transposition \( (12) \) acts by swapping \( x \) and \( y \). Hence the quotient near this orbit is locally 
    \[
    \mathbb{C}^n_u \times \textrm{Cone}(\mathbb{RP}^{2n-1})
    \] 
    and it is well-know the cone of $\mathbb{RP}^{d}$ when $d\geq 2$ is not a smooth manifold.  




\end{proof}




\begin{problem}
    
Let $C$ be the projective curve in $\mathbb{P}^2$ whose affine equation is 
\[y^2=\prod^{50}_{i=1}(x-i)^i\]
Characterize its singularities and calculate its arithmetic genus and its geometric genus (= the genus of its
desingularization).
\end{problem}
\begin{proof}
    The degree of the curve is $d=1275$, so its arithmetic genus is 
    \[g=\frac{(d-1)(d-2)}{2}=810901\]

Locally, the singularity corresponding to $y^2=(x-i)^i$ is cuspidal if $i$ is $i$ is odd, and nodal if $i$ is even. There is also a singularity at the point at infinity.


After normalization, $\tilde{C}$ is a double cover of $\mathbb{P}^1$, with branch points at the odd exponents and point at infinity, each with ramification index $2$. The Riemann-Hurwitz formula then says the geometric genus of $C$ is $12$. 


\end{proof}





\begin{problem}
    
The dual of a plane curve $C\subset \mathbb{P}^2$ of degree $d$ is the curve $C^{\vee}\subset (\mathbb{P}^2)^{\vee}$  consisting of all lines meeting $C$ in fewer than $d$ distinct points. For $C$ non-singular, this is simply the family of tangent lines to $C$. Show that for most non-singular curves $C$ of a given degree $d$, the only singularities of
the dual curve are nodes and cusps. For such $C$, determine the degree and number of nodes and
cusps of the dual curve. Show that if $d >2$ there are non-singular $C$ whose dual has singularities
worse than nodes and cusps. What happens if C is singular?



\end{problem}
The dual of a plane curve $C\subset\mathbb P^2$ of degree $d$ is the curve $C^{\vee}\subset(\mathbb P^2)^{\vee}$ consisting of all lines meeting $C$ in fewer than $d$ distinct points. For $C$ non‑singular, this is simply the family of tangent lines to $C$. Show that for most non‑singular curves $C$ of a given degree $d$, the only singularities of the dual curve are nodes and cusps. For such $C$, determine the degree and number of nodes and cusps of the dual curve. Show that if $d>2$ there are non‑singular $C$ whose dual has singularities worse than nodes and cusps. What happens if $C$ is singular?

\begin{proof}
For a smooth $C$ defined by a homogeneous polynomial $F(x,y,z)$, we have the Gauss map 
\[\gamma: C\to (\mathbb{P}^2)^{\vee}\]
given by 
\[p\mapsto [F_x(p): F_y(p):F_z(p)] \]

and the dual curve is the image of the Gauss map. If the map is a injective smooth immersion, then the image is smooth. Thus, the generic singularities of the dual curve can be either nodal, where the map is not injective, such that two (non-inflection) points on $C$ share a bitangent line; or the image of an inflection point where the Hessian vanishes simply, we have a cusp. The degree of the determinant of the Hessian is $3(d-2)$. 

For a curve whose dual having worse singularities, consider 
\[
F(x,y,z) = y^2z^2-x^4+x^2z^2
\]
In this case, the bitangent line corresponding to $2$ branches in the dual meeting no longer transversely, which gives rise to a tacnode.


\end{proof}


\section{Homework 4}

\begin{problem}[Hyperelliptic curves]
    Let $C$ be a hyperelliptic curve of genus $g$. 
    Is the natural multiplication map
    \[
      H^0(C, K_C)\otimes H^0(C,K_C)\longrightarrow H^0(C, K_C^{\otimes 2})
    \]
    surjective? What is the dimension of its image?
    \end{problem}
\begin{proof}
A standard basis of holomorphic differentials on \(C\) is
    \[
    \omega_i=\frac{x^i\,dx}{y}\qquad (i=0,1,\dots,g-1),
    \]
    so \(h^0(C,K_C)=g\).
    The products are
    \[
    \omega_i\omega_j=\frac{x^{i+j}(dx)^2}{y^2}
    =\;x^{i+j}\frac{(dx)^2}{f(x)},
    \]
    hence the image of the multiplication map is spanned by the sections
    \[
    x^k\frac{(dx)^2}{y^2}\qquad (k=0,1,\dots,2g-2).
    \]
    There are \(2g-1\) such monomials.


   \[ \dim Im(\big(H^0(K_C)\otimes H^0(K_C)\to H^0(K_C^{\otimes2})\big)=2g-1.
\]
In particular, the map is not surjective for \(g\ge3\) (since 
\(\dim H^0(K_C^{\otimes2})=3g-3\) by Riemann-Roch), while for \(g=2\) 
(and trivially for \(g=1\)) it is surjective.

\end{proof}



    
    \begin{problem}[Canonical map]
    Let $C$ be a curve of genus $g$ for which the map
    \[
      H^0(C, K_C)\otimes H^0(C,K_C)\longrightarrow H^0(C, K_C^{\otimes 2})
    \]
    is surjective, and let $\varphi\colon C\to \mathbb{P}^{\,g-1}$ be the canonical map.
    Use Riemann--Roch to calculate the dimension of the linear system of quadrics in
    $\mathbb{P}^{\,g-1}$ containing the image $\varphi(C)$. What is the base locus of
    this linear system? Does it depend continuously on the curve $C$? What happens if
    $C$ is trigonal? What happens if $C$ is hyperelliptic?
    \end{problem}
\begin{proof}

The space of quadrics in \(\mathbb P^{g-1}\) containing \(\varphi(C)\) is the kernel of
\[
Sym^2 H^0(C,K_C)\xrightarrow{\ \mu\ } H^0(C,K_C^{\otimes2}).
\]
Since \(dim Sym^2 H^0(C,K_C)=\binom{g+1}{2}=\tfrac{g(g+1)}{2}\) and by Riemann--Roch 
\(\dim H^0(C,K_C^{\otimes2})=3g-3\), we obtain
\[
dim Ker(\mu)=\frac{g(g+1)}{2}-(3g-3)
=\frac{g^2-5g+6}{2}
=\frac{(g-2)(g-3)}{2}.
\]
Hence the linear system of quadrics containing \(\varphi(C)\) has (projective) dimension 
\(\tfrac{(g-2)(g-3)}{2}-1\).


If \(C\) is trigonal, then \(\varphi(C)\) lies on a rational normal scroll \(S\subset \mathbb{P}^{g-1}\), and every quadric containing \(\varphi(C)\) also contains \(S\); thus the base locus is \(S\), which strictly contains the canonical curve.  
The same phenomenon occurs for the plane quintic curve in genus \(6\).

If \(C\) is hyperelliptic, the map \(\mu\) is not surjective for \(g\ge3\).  
In this case the canonical map factors through the hyperelliptic involution and its image is a rational normal curve in \( \mathbb{P}^{g-1}\).  
The quadrics containing \(\varphi(C)\) are precisely those containing this rational normal curve, and the base locus is that curve.







\end{proof}



    
    \begin{problem}[Abel map]
    Consider the Abel map $M_g \to A_g$. Is it an immersion at
    hyperelliptic curves? Is it an immersion at curves as in Problem~2?
    \end{problem}
    \begin{proof}
        Let $[C]\in M_g$.  There are natural identifications
\[
T_{[C]}M_g \cong H^1(C,T_C)\cong H^0(C,K_C^{\otimes2})^\vee,
\qquad
T_{[J(C)]}A_g \cong Sym^2 H^0(C,K_C)^\vee.
\]
The differential of the Abel map
\[
d\tau\colon T_{[C]}M_g\longrightarrow T_{[J(C)]}A_g
\]
is, up to these identifications, the dual of the multiplication map
\[
\mu:\;Sym^2 H^0(C,K_C)\longrightarrow H^0(C,K_C^{\otimes2}).
\]
Thus \(d\tau\) is injective, i.e.\ the Abel map is an immersion at \(C\),
if and only if \(\mu\) is surjective.

For a hyperelliptic curve of genus \(g\ge3\), the image of \(\mu\) has
dimension \(2g-1<3g-3\), so \(\mu\) is not surjective.  Therefore \(d\tau\)
has kernel of dimension \(g-2\), and the Abel map is not an immersion at
hyperelliptic curves.

For a nonhyperelliptic curve \(C\), Petri’s theorem implies that \(\mu\)
is surjective except in the classical exceptional cases: trigonal curves
and the plane quintic (in genus \(6\)).  Hence the Abel map is an immersion
at curves \(C\) for which the multiplication map is surjective—namely,
for the general nonhyperelliptic curve—but fails to be an immersion at
hyperelliptic and trigonal curves.
    \end{proof}
    
    \begin{problem}[Fubini--Study pullbacks]
    \begin{enumerate}
      \item[(a)] Show that restricting the Fubini--Study K\"ahler form $\omega_{\mathrm{FS}}$
      on $\mathbb{P}^n$ to the standard hyperplane $\mathbb{P}^{n-1}\subset\mathbb{P}^n$
      yields the Fubini--Study form on $\mathbb{P}^{n-1}$.
      \item[(b)] Let $A\in\mathrm{PGL}(n+1,\mathbb{C})$, and let $F_A\colon\mathbb{P}^n\to\mathbb{P}^n$
      be the induced automorphism. Show that $F_A^*\omega_{\mathrm{FS}}=\omega_{\mathrm{FS}}$
      if and only if $A$ is in the image of $U(n+1)$.
    \end{enumerate}
    \end{problem}
\begin{proof}
        Write homogeneous coordinates \([Z_0:\dots:Z_n]\) on \(\mathbb{P}^n\) and take the standard affine chart 
        \(U_0=\{Z_0\neq0\}\) with inhomogeneous coordinates \(z_i=Z_i/Z_0\) for \(i=1,\dots,n\).
        On \(U_0\) the Fubini--Study form has the local potential
        \[
        \Phi(z)=\log\!\big(1+\sum_{i=1}^n|z_i|^2\big),
        \qquad 
        \omega_{\mathrm{FS}}=\frac{i}{2\pi}\partial\bar\partial\Phi.
        \]
        The standard hyperplane \(\mathbb{P}^{n-1}\subset\mathbb{P}^n\) may be taken as \(Z_n=0\). 
        Its intersection with \(U_0\) is given by \(z_n=0\), and the induced inhomogeneous coordinates on this chart of 
        \(\mathbb{P}^{n-1}\) are \(z_1,\dots,z_{n-1}\). Restricting \(\Phi\) to \(z_n=0\) gives
        \[
        \Phi|_{z_n=0}=\log\!\big(1+\sum_{i=1}^{n-1}|z_i|^2\big),
        \]
        which is exactly the Fubini--Study potential on this chart of \(\mathbb{P}^{n-1}\).
        Since the Kähler form is \(\frac{i}{2\pi}\partial\bar\partial\) of the potential, 
        the restriction \(\omega_{\mathrm{FS}}|_{\mathbb{P}^{n-1}}\) equals the Fubini--Study form of \(\mathbb{P}^{n-1}\). 

        Let \(M\in\mathrm{GL}(n+1,\mathbb{C})\) be any lift of \(A\in\mathrm{PGL}(n+1,\mathbb{C})\),
        and let \(F_M\) be the induced map on \(\mathbb{P}^n\).
        On the affine chart \(U_0\) one may represent points by homogeneous vectors \(Z=(1,z_1,\dots,z_n)^T\).
        The pullback of the Fubini--Study potential under the linear map \(Z\mapsto MZ\) is
        \[
        \Phi\circ F_M([Z])=\log\big(Z^*M^*M Z\big),
        \]
        while the original potential is \(\log(Z^*Z)\). Thus
        \[
        F_M^*\omega_{\mathrm{FS}}=\omega_{\mathrm{FS}}
        \iff 
        \partial\bar\partial\big(\log(Z^*M^*M Z)-\log(Z^*Z)\big)=0.
        \]
        The function
        \[
        h([Z])=\log(Z^*M^*M Z)-\log(Z^*Z)
        \]
        is homogeneous of degree \(0\), hence well-defined on \(\mathbb{P}^n\). 
        The above condition means that \(h\) is pluriharmonic on the compact manifold \(\mathbb{P}^n\), 
        so \(h\) must be constant: \(h([Z])=\log c\) for some \(c>0\). 
        Thus for all nonzero \(Z\in\mathbb{C}^{n+1}\),
        \[
        Z^*M^*M Z = c\,Z^*Z.
        \]
        Polarizing this identity gives \(M^*M=cI\). Hence \(M=\sqrt{c}\,U\) for some \(U\in U(n+1)\).
        Conversely, if \(M=\lambda U\) with \(U\) unitary and \(\lambda\in\mathbb{C}^*\), then
        \[
        \log(Z^*M^*M Z)=\log(|\lambda|^2 Z^*Z)=\log|\lambda|^2+\log(Z^*Z),
        \]
        so \(\partial\bar\partial\) of the difference vanishes, and \(F_M^*\omega_{\mathrm{FS}}=\omega_{\mathrm{FS}}\).
        
        Therefore \(F_A^*\omega_{\mathrm{FS}}=\omega_{\mathrm{FS}}\) if and only if any (and hence every) lift \(M\) of \(A\) 
        satisfies \(M^*M=cI\), that is, if and only if \(A\) is represented in \(\mathrm{PGL}(n+1,\mathbb{C})\) 
        by a scalar multiple of a unitary matrix.  Equivalently, \(A\) lies in the image of the natural map 
        \(U(n+1)\to\mathrm{PGL}(n+1,\mathbb{C})\).
\end{proof}
    
    \begin{problem}[Harmonicity of the K\"ahler form]
    Let $(X,g)$ be a K\"ahler manifold with
    K\"ahler form $\omega$. Show that $\omega$ is harmonic with respect to the Laplacian
    $\Delta = d d^{*} + d^{*} d$.
    \end{problem}
    
\begin{proof}

The Kähler form $\omega$ is closed. With the Kähler identities and using the property that $\omega$ is of type $(1,1)$ and covariantly constant with respect to the Levi-Civita connection, one finds
\[
d^*\omega = 0.
\]

Since $\omega$ is both closed and co-closed, it is harmonic:
\[
\Delta \omega = (d d^* + d^* d)\,\omega = 0.
\]

Hence the Kähler form $\omega$ is harmonic with respect to the Laplacian.
\end{proof}

\section{Homework 5}
\begin{problem}
    Let $S$ be a non-singular quadric suface. We saw that $S\cong \mathbb{P}^1\times \mathbb{P}^1$. For $C\subset S$ a non-singular curve, define its bidegree $(d,e)$ by intersecting $C$ with the lines $pt \times \mathbb{P}^1$ and $\mathbb{P}^1 \times pt$. Compute the genus of $C$ in terms of $d,e$. 
\end{problem}
\begin{proof}
    We wish to use the adjunction formula 
    \[\omega_C\cong i^*(\omega_S\otimes \mathcal{O}(C))\]
Let $H_1,H_2$ denote the hyperplane section corresponding to $O(1)$ on the two copies of $\mathbb{P}^1$. The canonical bundle $\omega_X$ is associated to the divisor $-2H_1-2H_2$, coming from pulling back $H_i$ along the projection maps. The curve $C$ is the divisor $dH_1+eH_2$. Thus by looking at the canonical divisors, we have 
\[K_C=K_S+C|_C=(-2H_1-2H_2+dH_1+eH_2)(dH_1+eH_2)\]
Taking degree (since $[H_i]\cap [H_i]$ is empty, and $H_1\cap H_2$ is a single point)
\[2-2g=(-2+d)e+(-2+e)d\]
Simplifying gives us 
$g=(d-1)(e-1)$.




\end{proof}





\begin{problem}
    Let $S$ be a non-singular cubic surface and $p\in S$. Show that $Bl_p(S)$ is a del Pezzo surface $dP2$.
\end{problem}
\begin{proof}
    A smooth cubic surface can be obtained by blowing up $6$ points in general position in $\mathbb{P}^2$, which is a del Pezzo surface of degree $3$.(This uses the next problem) Blowing up a seventh point gives you a del Pezzo surface. The degree reduces by $1$, since the caonical divisor of the blow up is the pullback of the canonical divisor plus the exception divisor, which has empty intersection. On the other hand, the exceptional divisor has self-intersection number $-1$ by identifying the normal bundle as $\mathcal{O}_{\mathbb{P}^1}(-1)$.
\end{proof}




\begin{problem}
    Prove that a non-singular cubic surface $S$ is rational, i.e there is a rational map $S\to \mathbb{P}^2$ which is an isomorphism away from some curves in S and $\mathbb{P}^2$. Hint: you may want ot prove first that $S$ contains two disjoint lines. 
\end{problem}
\begin{proof}
    Is suffices to demonstrate $S$ as the blow-up of $\mathbb{P}^2$ at $6$ points, since blowup morphism is a birational equivalence. Assume the surface contains two disjoint lines $L_1,L_2$. We have a birational equivalence 
    \[f:\mathbb{P}^1\times \mathbb{P}^1\to S\] 
    given by mapping a pair of points $(s,t)$ on the two lines to the intersection point of the surface with the line spanned by $s,t$. Its inverse 
    \[g: S\setminus L_1\cup L_2 \mathbb{P}^1\times \mathbb{P}^1\]
    is given by mapping a point $s$ not on either line to the pair of points $(L_1\cap \textrm{span}(s,L_2),L_2\cap \textrm{span}(s,L_1))$. The fives lines meeting both $L_1$ and $L_2$ are blown-down, realizing $S$ as the blowup of $\mathbb{P}^1\times \mathbb{P}^1$ as the blow up of $5$ points, which is same as blowing up $\mathbb{P}^2$ at $6$ points.  





\end{proof}

\begin{problem}
    Compute the degrees of the Veronese and Segre varieties. 
\end{problem}
\begin{proof}
    The Veronese embedding is given by the line bundle $\mathcal{O}_{\mathbb{P}^n}(d)$ and its canonical linear system. The pullback of the hyperplane section on $\mathbb{P}^N$ is a degree $d$ hypersurface. The degree of the Veronese varieties is the intersection of $n$ degree $d$ hypersurfaces in $\mathbb{P}^n$, which is $d^n$.

    For the Segre embedding, the canonical hyperplane section pullback to the sum $H_1+H_2$, which are the canonical hyperplane sections of $\mathcal{O}(1)$ on each copy of the projective space $\mathbb{P}^m$ and $\mathbb{P}^n$. Intersecting $H_1+H_2$ $m+n$ times gives $\binom{m+n}{n}$ intersection points.


\end{proof}













\end{document}