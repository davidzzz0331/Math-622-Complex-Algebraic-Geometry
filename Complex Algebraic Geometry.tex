\documentclass{article}
\usepackage[utf8]{inputenc}
\usepackage{amsmath}
\usepackage{amsfonts}
\usepackage{amssymb}
\usepackage{tikz}
\usepackage{fullpage}
\usepackage{tikz-cd}
\usepackage{spectralsequences}
\usepackage{adjustbox}
\usepackage{xfrac}
\usepackage{tcolorbox}
\usepackage{xcolor}
\usepackage{hyperref}
\usepackage{graphicx}
\graphicspath{ {D:/Chrome Downloads./} }
\usepackage[parfill]{parskip}
\usepackage{amsthm}
\usetikzlibrary{calc}
\theoremstyle{definition}
\newtheorem{theorem}{Theorem}[section]
\theoremstyle{definition}
\newtheorem{definition}{Definition}[theorem]
\theoremstyle{definition}
\newtheorem{remark}{Remark}[theorem]
\theoremstyle{definition}
\newtheorem{proposition}{Proposition}[theorem]
\theoremstyle{definition}
\newtheorem{lemma}[theorem]{Lemma}
\theoremstyle{definition}
\newtheorem{corollary}{Corollary}[theorem]
\theoremstyle{definition}
\newtheorem{example}{Example}[theorem]
\tikzset{curve/.style={settings={#1},to path={(\tikztostart)
    .. controls ($(\tikztostart)!\pv{pos}!(\tikztotarget)!\pv{height}!270:(\tikztotarget)$)
    and ($(\tikztostart)!1-\pv{pos}!(\tikztotarget)!\pv{height}!270:(\tikztotarget)$)
    .. (\tikztotarget)\tikztonodes}},
    settings/.code={\tikzset{quiver/.cd,#1}
        \def\pv##1{\pgfkeysvalueof{/tikz/quiver/##1}}},
    quiver/.cd,pos/.initial=0.35,height/.initial=0}
\title{Math 622: Complex Algebraic Geomtry}
\author{David Zhu}

\begin{document}
\maketitle

This is notes taken for Math 622: Complex Algebraic Geometry in Fall 2025, taught by Professor Ron Donagi at UPenn. 

\section{Overview of Hodge Theory}
\subsection{Wed 27}
The first part of the course will be a primer on complex algebraic geometry, in particular Hodge theory, without too much detailed proofs. 


\begin{tcolorbox}[colback=purple!5!white,colframe=purple!75!black]
\begin{definition}
A \textbf{Kahler manifold} is a complex maniolfd with a Hermitian metric whose imaginary part is closed. 
\end{definition}
\end{tcolorbox}


\begin{tcolorbox}[colback=yellow!5!white,colframe=yellow!30!white]
\begin{example}
The complex projective spaces $\mathbb{CP}^n$ with Fubini-Study metric are Kahler manifolds. Submanifolds of projective space inherit Kahler structures.
\end{example}
\end{tcolorbox}
A reference for this is \cite{Huy} 3.1.9.






\begin{tcolorbox}[colback=red!5!white,colframe=red!30!white]
\begin{theorem}[Kodaira Embedding]
A compact complex maniolf dadmits a holomorphic embedding into projective space iff it has a Kahler metric whose Kahler form is integral.
\end{theorem}
\end{tcolorbox}
The upshot is that Hodge and Lefshetz decomositions work for compact Kahler manifolds. 

Review of complex differential forms


\begin{tcolorbox}[colback=purple!5!white,colframe=purple!75!black]
\begin{definition}[Complex Differential Forms]

\end{definition}
\end{tcolorbox}





\begin{tcolorbox}[colback=purple!5!white,colframe=purple!75!black]
\begin{definition}
For a complex manifold $X$, we have a decomposition of differential forms
\[\Omega^k_{X,\mathbb{C}}:=\oplus_{p+q=k}\Omega^{p,q}_{X}\]
\end{definition}
\end{tcolorbox}



\begin{tcolorbox}[colback=red!5!white,colframe=red!30!white]
\begin{theorem}[Hodge Decomposition]
For a compact Kahler manifold
\[H^{k}(X,\mathbb{C})\cong \oplus_{p+q=k}H^{p,q}(X)\]
and 
\[H^{p,q}(X)\cong \overline{H^{p,q}(X)}\]
\end{theorem}
\end{tcolorbox}
The left hand side will be the DeRham cohomology, and the left hand side is the Dolbeaut cohomoology. 

\begin{tcolorbox}[colback=yellow!5!white,colframe=yellow!30!white]
\begin{example}[Hodge Diamond of a Riemann Surface]
    For a compact Riemann surafce $X$ of genus $g$, the Hodge numbers are 
    \[h^{0,0}=1, h^{1,0}=g, h^{0,1}=g, h^{1,1}=1\]


    \[\begin{tikzcd}
        & 1 \\
        g && g \\
        & 1
    \end{tikzcd}\]
\end{example}
\end{tcolorbox}


\begin{tcolorbox}[colback=yellow!5!white,colframe=yellow!30!white]
\begin{example}
For a compact complex surface $X$, the Hodge numbers are 
\[\begin{tikzcd}
	&& {h^{0,0}} \\
	& {h^{1,0}} & {} & {h^{0,1}} && {} \\
	{h^{2,0}} && {h^{1,1}} && {h^{0,2}} \\
	& {h^{2,1}} && {h^{1,2}} \\
	&& {h^{2,2}}
\end{tikzcd}\]
\end{example}
\end{tcolorbox}


\begin{tcolorbox}[colback=purple!5!white,colframe=purple!75!black]
\begin{definition}
Given the Kahler form $\omega$. define the Lefshetz operator 
\[L: H^{k}(X, \mathbb{R})\to H^{k+2}(X, \mathbb{R}) \]
given by 
\[\alpha\mapsto \omega \wedge \alpha\]
\end{definition}
\end{tcolorbox}


\begin{tcolorbox}[colback=red!5!white,colframe=red!30!white]
\begin{theorem}[Hard Lefshetz]
    For $n= dim X$, and $k\leq n$, we have 
    \[L^{n-k}: H^{k}(X, \mathbb{R})\cong H^{2n-k}(X, \mathbb{R})\]

\end{theorem}
\end{tcolorbox}


\begin{tcolorbox}[colback=purple!5!white,colframe=purple!75!black]
\begin{definition}[Primitive Cohomology]
    \[H^{k}(X, \mathbb{R})_{\textrm{prim}}:= ker(L^{n-k+1})\]
\end{definition}
\end{tcolorbox}


\subsection{Harmonic Forms}
\begin{enumerate}
    \item Equip $X$ with a Kahler metric. 
    \item Define the adjoint $d^*$ of $d$, and the Laplacian 
    \[\Delta_{d}=dd^*+d^*d\]
    \item The harmonic forms are 
    \[\mathcal{H}^k{X}:= ker \Delta_d\]
\end{enumerate}



\begin{tcolorbox}[colback=green!5!white,colframe=green!30!white]
\begin{corollary}[Harmonic Decomposition]
If $\alpha$ is harmonic, then each $(p,q)$-component of $\alpha$ is harmonic,. hnece 
\[\mathcal{H}(X)\cong \oplus_{p+q=k} \mathcal{H}^{p,q}(X)\]
\end{corollary}
\end{tcolorbox}

The Frolicher spectra seauence is obtained from the Hodge filtratiokn on the de Rham complex. 


\begin{tcolorbox}[colback=red!5!white,colframe=red!30!white]
\begin{theorem}
For a compact Kahler manifold, the Frolicher spectral sequence degenerates at $E_1$.
\end{theorem}
\end{tcolorbox}
The theorem implies the Hodge filtration on cohomology 
\[F^pH^k(X, \mathbb{C})= \oplus_{r\geq p}H^{r,k-r}(X)\]


\begin{tcolorbox}[colback=purple!5!white,colframe=purple!75!black]
\begin{definition}
A \textbf{Hodge structure} of \textbf{weight} $k$ is a decomposition 
\[H_{\mathbb{C}}=\oplus_{p+1=k}H^{p,q}\]
with $H^{p,q}=\overline{H^{p,q}}$
\end{definition}
\end{tcolorbox}

A story is that there is a dense open subset in $\mathbb{P}^9$ that parameterizes cubic curves in $\mathbb{P}^3$. Replacing the curves with their cohomology gives you a vector bundle over the base space. 



\begin{tcolorbox}[colback=purple!5!white,colframe=purple!75!black]
\begin{definition}
An \textbf{analytic cycle} of codimension $k$ is a linear combination of irreducible analytic subset of codimension $k$.
\end{definition}
\end{tcolorbox}

The Hodge conjecture states that if 
\[\alpha \in H^{2k}(X, \mathbb{Q})\cap H^{k,k}(X)\], then $\alpha$ is an algebraic cycle. 

\section{Wed, Sep 3}
Did not attend LOL. But here is what I think is covered: 

We begin with the definition of a complex manifold.
\begin{tcolorbox}[colback=purple!5!white,colframe=purple!75!black]
\begin{definition}
A \textbf{holomorphic atlas} on a smooth manifold is an atlas $\{ U_i, \varphi_i \}$ of the form 
\[\varphi_i: U_i\cong \varphi_i(U_i)\subset \mathbb{C}^n\]
where the $\cong$ sign denotes a homeomorphism. Two atlases are considered equivalent if for every $(U_i,\varphi_i)$ and $(U'_j,\varphi'_j)$, the maps 
\[\varphi_i\circ \varphi'^{-1}_j: \varphi'_j(U_i\cap U'_j)\to \varphi_i(U_i\cap U'_j)\]

are holomophic.
\end{definition}
\end{tcolorbox}


\begin{tcolorbox}[colback=purple!5!white,colframe=purple!75!black]
\begin{definition}
A \textbf{complex manifold of dimension} $n$ is a smooth manifold of dimension $2n$ endowed with an equivalence class of holomorphic atlas.  
\end{definition}
\end{tcolorbox}

One difference between smooth manifold and complex manifolds is: every open subset of $\mathbb{R}^n$ contains a open ball diffeomorphic to $\mathbb{R}^n$. Thus, every smooth manifold can be covered by charts diffeomorphic to $\mathbb{R}^n$. However, in general a complex manifold cannot be covered by open subsets biholomorphic to $\mathbb{C}^n$. This is because $\mathbb{C}$ is not biholomorphic to any of its proper open subsets. Considering the basis for the topology, one can say complex manifolds are modelled after the complex disks $\mathbb{D}^n$.


\begin{tcolorbox}[colback=purple!5!white,colframe=purple!75!black]
\begin{definition}
A \textbf{homolomorphic function} on complex manifold $X$ is a function 
\[f: X\to \mathbb{C}\]
such that for all charts $\varphi: U_i\to \mathbb{C}$, the maps 
\[f\circ \varphi_i^{-1}: \varphi_i(U_i)\to \mathbb{C} \]
is holomorphic. 
\end{definition}
\end{tcolorbox}

From the maximum modulus principle for several variables, one can deduce the following version of Liouville's theorem 


\begin{tcolorbox}[colback=blue!5!white,colframe=blue!30!white]
\begin{proposition}
Any global holomorphic function on a connected compact complex manifold $X$ is constant. 
\end{proposition}
\end{tcolorbox}

In particular, this means there is no ``partition of unity'' for complex manifolds, where we would dream of gluing holomorphic functions together like we could for smooth manifolds. Thus, complex manifolds are much more rigid. 


\begin{tcolorbox}[colback=purple!5!white,colframe=purple!75!black]
\begin{definition}
Let $X$ be a complex manifold. The \textbf{sheaf of holomorphic functions} on $X$, denoted by $\mathcal{O}_X$, is defined by 
\[\mathcal{O}_X(U):= \{ f: U\to \mathbb{C}: f \textrm{ holomorphic} \}\] 
\end{definition}
\end{tcolorbox}
The sheaf conditions are easy to check. By the previous proposition, it is immediate that if $X$ is compact and connected, 
\[\Gamma(X, \mathcal{O}_X)= \mathbb{C}\]. 

Here is a famous theorem in complex analysis showing the compactness can be slightly weakened


\begin{tcolorbox}[colback=red!5!white,colframe=red!30!white]
\begin{theorem}[Hartog's Theorem]
Let $G$ be an open subset of $\mathbb{C}^n$, where $n\geq 2$. Let $K\subset G$ be a compact subset. If $G\setminus K$ is connected, then any holomorphic function on $G\setminus K$ can be extended over to the entire $G$.  
\end{theorem}
\end{tcolorbox}
In particular, for $X$ a complex manifold of dimension at least $2$, we have
\[\Gamma(X, \mathcal{O}_X)\cong \Gamma(X-\{ x \}, \mathcal{O}_X )\]

\begin{tcolorbox}[colback=green!5!white,colframe=green!30!white]
\begin{remark}
Here is an example of the failure of Hartog's theorem in dimension $1$: $f(z)=\frac{1}{z}$ is holomorphic on the punctured plane, but cannot be extended to the entire $\mathbb{C}$. 
\end{remark}
\end{tcolorbox}


\begin{tcolorbox}[colback=purple!5!white,colframe=purple!75!black]
\begin{definition}
A continous map $f: X\to Y$ between complex manifolds is \textbf{holomorphic} the corresponding atlas maps are holomorphic. 
\end{definition}
\end{tcolorbox}



\begin{tcolorbox}[colback=purple!5!white,colframe=purple!75!black]
\begin{definition}
A \textbf{meromorphic function} on a complex manifold is a map
\[f: X\to \coprod_{x\in X} \textrm{Quot}( \mathcal{O}_{X,x})\]
which associates every point $x\in X$ an element $f_x\in \textrm{Quot}(\mathcal{O}_{X,x})$. Moreover, $f_x$ are required to be locally a quotient of holomorphic functions. The \textbf{sheaf of meromorphic functions} is then denoted by $\mathcal{K}_X$, and its global sections $K(X)$. 
\end{definition}
\end{tcolorbox}

Here is a theorem that bounded the size of meromorphic functions on a complex manifold. 
\begin{tcolorbox}[colback=red!5!white,colframe=red!30!white]
\begin{theorem}[Siegel's Theorem]
Let $X$ be a compact connected complex manifold of dimension $n$. Then 
\[\textrm{Trdeg}_{\mathbb{C}}K(X)\leq n\]
\end{theorem}
\end{tcolorbox}


\begin{tcolorbox}[colback=purple!5!white,colframe=purple!75!black]
\begin{definition}
The \textbf{algebraic dimension} of a compact connect complex manifold $X$ is 
\[a(X):= \textrm{Trdeg}_{\mathbb{C}}(X)\]
\end{definition}
\end{tcolorbox}
Note that the definition works properly only for compact manifold. For example, if we consider the function field of $\mathbb{C}$, it is much larger than its algebraic counter part, since 
\[f(z)=\frac{1}{\textrm{sin}(z)}\]
is a meromorphic function on $\mathbb{C}$. 

\begin{tcolorbox}[colback=yellow!5!white,colframe=yellow!30!white]
\begin{example}
The algebraic dimension of $\mathbb{P}^n$ is $n$.
\end{example}
\end{tcolorbox}

\begin{proof}
    For $\mathbb{P}^2$: we claim the algebraic dimension for $\mathbb{P}^2$ is $2$. From Siegel's theorem, it suffices to show $a(\mathbb{P}^2)\geq 2$. Define 
    \[f_i: \mathbb{P}^2\to \coprod_{x\in \mathbb{P}^2}Q(\mathcal{O}_{\mathbb{P}^{2},x} )\]
    by sending a point $p$ to the germ of the rational function $\frac{z_i}{z_0}\in Q(\mathcal{O}_{\mathbb{P}^2,p})$ for $i=1,2$. Locally on $U_0\cong \mathbb{A}^2=\{ (u,v):=(\frac{z_1}{z_0}, \frac{z_2}{z_0})\}$, the assignment $f_i$ are just the coordinate functions; on $U_1\cong \mathbb{A}^2=\{ (u,v):=(\frac{z_0}{z_1}, \frac{z_2}{z_1})\}$, we have $f_1=\frac{1}{u}$ and $f_2= \frac{v}{u}$. The change of coordinates with $U_2$ is similar. Clearly, $\frac{z_1}{z_0}$ and $\frac{z_2}{z_0}$ do not satisfy any algebraic relations by degree reasons, so the function field is of trascendence degree at least $2$. 

\end{proof}



\subsection{Examples of Complex Manifolds }
Of course, $\mathbb{C}^n$ and $\mathbb{CP}^n$ are well-known examples of complex manifolds. Their complex structure is very easy to check. 




\begin{tcolorbox}[colback=yellow!5!white,colframe=yellow!30!white]
\begin{example}[Quotients]
Let $X$ be a complex manifold and $G$ a complex Lie group acting properly discontinuously on $X$. Then the quotient $X/G$ is a complex manifold such that the quotient map 
\[\pi: X\to X/G\]
is holomorphic.
\end{example}
\end{tcolorbox}
By acting properly discontinuously, we ensure $X/G$ is Hausdorff and admits holomorphic charts: we may find an open covering $\{ U_i\}$ of $X$ such that $g(U_i)\cap U_i= \emptyset$ for $g\neq 1$. The image of the $U_i$ under the projection map gives us charts on $X/G$, since 
\[U_i\to \pi(U_i)\]
is a homeomorphism. 

An important class of complex manifolds arising from such quotients is complex tori. 


\begin{tcolorbox}[colback=yellow!5!white,colframe=yellow!30!white]
\begin{example}
Let $V= \mathbb{C}^n$ and $\Gamma\subset V$ be a discrete subgroup under addition that generates $V$. Then, $\Gamma$ must be freely generated by a real basis of $V$, which is of order $2n$. Then, $\Gamma$ acts properly discontinuously on $V$ by translation. The quotient 
\[\mathbb{C}^n/\Gamma\]
is an $n$-dimensional \textbf{complex torus}. Topologically they are homeomorphic to a real torus of dimension $2n$. However, two different lattices $\Gamma_1$ and $\Gamma_2$ can induce non-isomorphic complex manifolds. The problem is that we must have a $\mathbb{C}$-linear automorphism on $\mathbb{C}^n$ that takes $\Gamma_1$ to $\Gamma_2$ to induce an isomorphism. 
\end{example}
\end{tcolorbox}
A one-dimensional complex torus is called an \textbf{elliptic curve}. By a change of coordinates, we may assume that the defining lattice is of the form $\mathbb{Z}+ \tau \mathbb{Z}$ for some $\tau \in \mathbb{H}$. It turns out that $\tau_1,\tau_2$ define isomorphic elliptic curves iff there is a matrix in $\textrm{SL}(2, \mathbb{Z})$ that takes one to the other. cn



\begin{tcolorbox}[colback=yellow!5!white,colframe=yellow!30!white]
\begin{example}
Let $f: \mathbb{C}^n\to \mathbb{C}$ be a holomorphic function, and $0$ is a regular value. Then, the locus 
\[V(f)\subset \mathbb{C}^n\]
is called an \textbf{affine hypersurface}, and by the implicit function theorem we get that $V(f)$ can be endowed with a complex structure. \newline

Similarly, \textbf{projective hypersurfaces}, which are vanishing sets of homeogeoneous plynomials in $\mathbb{P}^n$, are complex manifolds. More generally, suppose $f_1,...,f_k: \mathbb{C}^n\to \mathbb{C}$ are holomorphic functions, and $0$ is a regular value of the function 
\[(f_1,...,f_k): \mathbb{C}^n\to \mathbb{C}^k\]

then the intersection $V(f_1)\cap...\cap V(f_k)$ is a complex manifold of dimension $n-k$. Manifolds arising this way are called \textbf{complete intersections}. 
\end{example}
\end{tcolorbox}














We spend the rest of the section discussing Grassmannians carefully. 


\begin{tcolorbox}[colback=purple!5!white,colframe=purple!75!black]
\begin{definition}
For $1\leq k\leq n$, the \textbf{Grassmannian} $\textrm{Gr}(n,k)$ is the set of $k$-dimensional linear subspaces of $\mathbb{C}^n$. 
\end{definition}
\end{tcolorbox}
One way to define a smooth manifold structure on the Grassmannians is via identification with projection operators. We now introduce the Plucker embedding to show that the Grassmannians are actually projective varieties. 


For any $k$-plane $V\subset \mathbb{C}^{n}$, we may choose any basis $v_1,...,v_k$ of $V$
 \begin{tcolorbox}[colback=purple!5!white,colframe=purple!75!black]
\begin{definition}
The \textbf{Plucker embedding} is the map 
\[i: \textrm{Gr}(k,n)\to \mathbb{P}(\bigwedge^k \mathbb{C}^n)\]
given by $V\mapsto [v_1\wedge ...\wedge v_k]$. 
\end{definition}
\end{tcolorbox}
Note that the wedge product of a different choice of basis vectors differs by the determinant of change of basis matrix, so the map is well-defined and injective. Suppose we are given by a basis $v_1,...,v_k$ of a $k$-plane $V$, which forms a $n\times k$ matrix $M_V$; the standard basis of $\bigwedge^k \mathbb{C}^n$ are given by basis $\{e_{i_1}\wedge ...\wedge e_{i_k}:1\leq i_1<...<i_k\leq n\}$. Thus, the coordinate of $v_1\wedge...\wedge v_k$, with respect to the standard basis vector indexed by $I$, is given by the $I$th $k\times k$ minors of $M_V$. 

The image of the Plucker embedding has to satisfy the Plucker relations, which is a set of quadratic polynomials relating the coordinates. It turns out that these quadratics precisely cut out the image of the Plucker embedding. A proof of the fact can be found in \cite{GH} Page 210.



















\section{Mon, Sep 8}
Review of complex manifolds and sheaves. 


\begin{tcolorbox}[colback=blue!5!white,colframe=blue!30!white]
\begin{proposition}
There is an equivalence between vector bundles and locally free $\mathcal{O}_X$-modules through sheaf of sections. 
\end{proposition}
\end{tcolorbox}


\begin{tcolorbox}[colback=purple!5!white,colframe=purple!75!black]
\begin{definition}
The \textbf{Picard group} of $X$ is the set of isomorphism classes of line bundles under tensor product. 
\[\textrm{Pic}(X)\cong H^1(X, \mathcal{O}^*_X)\]
\end{definition}
\end{tcolorbox}
The exponential sequence 
\[\begin{tikzcd}
0\arrow[r]&\mathbb{Z}\arrow[r]& \mathcal{O}_X\arrow[r]&\mathcal{O}^*_X\arrow[r]&0
\end{tikzcd}\]
induces the map 
\[c_1: \textrm{Pic}(X)\to H^2\]


\begin{tcolorbox}[colback=purple!5!white,colframe=purple!75!black]
\begin{definition}[\textbf{Canonical ring}]
We define
\[R(X)=\oplus_{m\geq 0} H^0(X,K_X^{\otimes m})\]
where $K_X$ is the canonical bundle. 
\end{definition}
\end{tcolorbox}


\begin{tcolorbox}[colback=purple!5!white,colframe=purple!75!black]
\begin{definition}
A sheaf $F$ is \textbf{flasque} is every restriction map 
\[F(X)\to F(U)\]
is surjective.
\end{definition}
\end{tcolorbox}


\begin{tcolorbox}[colback=green!5!white,colframe=green!30!white]
\begin{remark}
Quote from Ron: A flasque sheaf is almost never ``nice'': for example, the sheaf of continuous functions of a punctured disk is not flasque.
\end{remark}
\end{tcolorbox}


\begin{tcolorbox}[colback=blue!5!white,colframe=blue!30!white]
\begin{proposition}
Every sheaf has a flasque resolution. 
\end{proposition}
\end{tcolorbox}


\begin{tcolorbox}[colback=purple!5!white,colframe=purple!75!black]
\begin{definition}
The \textbf{sheaf cohomology} of a space with a sheaf $(M,F)$ is the cohomology of the chain complex obtained from takning global sections of any flasque resolution.  
\end{definition}
\end{tcolorbox}
As usual, it is independent of the chosen resolution, and a short exact sequence of sheaves induces a long exact sequence on cohomology. 


\begin{tcolorbox}[colback=purple!5!white,colframe=purple!75!black]
\begin{definition}
A sheaf is \textbf{acyclic} if all higher cohomologies are $0$. 
\end{definition}
\end{tcolorbox}


\begin{tcolorbox}[colback=red!5!white,colframe=red!30!white]
\begin{theorem}
For paracompact spaces, the Cech cohomology and sheaf cohomology are naturally isomorphic. 
\end{theorem}
\end{tcolorbox}


\begin{tcolorbox}[colback=blue!5!white,colframe=blue!30!white]
\begin{proposition}
For smooth line bundles, 
\[c_1: \textrm{Pic}(X)\to H^2(X; \mathbb{Z})\]
is an isomorphism, for $\mathcal{O}^*_X$ is \textbf{soft}, thus acyclic.
\end{proposition}
\end{tcolorbox}
For holomorphic case, this is \textbf{NOT} true. 


\begin{tcolorbox}[colback=blue!5!white,colframe=blue!30!white]
\begin{proposition}[Global Sections of $\mathcal{O}(k)$]
For $k>0$, there is an isomorphism 
\[H^0(\mathbb{P}^n, \mathcal{O}(k))\cong \mathbb{C}[z_0,...,z_n]_k\cong \textrm{Sym}^k*(V^{\vee})\]
\end{proposition}
\end{tcolorbox}



Let $T_{\mathbb{P}^n}$ and $\Omega^1_{\mathbb{P}^n}$ be the holomorphic tangent and cotangent bundles. 


\begin{tcolorbox}[colback=blue!5!white,colframe=blue!30!white]
\begin{proposition}[Euler Sequence]
On $\mathbb{P}^n$ there is a natural short exact sequence of holomorphic vector bundles 
\[\begin{tikzcd}
0\arrow[r]&\mathcal{O}_{\mathbb{P}^n}\arrow[r]& \mathcal{O}_{\mathbb{P}^n}(1)^{\oplus (n+1)}\arrow[r]& T_{\mathbb{P}^n}\arrow[r]&0
\end{tikzcd}\]

Alternatively, one may dualize and get 
\[ \begin{tikzcd}
0\arrow[r]&\Omega^1_{\mathbb{P}^n}\arrow[r]& \mathcal{O}_{\mathbb{P}^n}(-1)^{\oplus (n+1)}\arrow[r]&\mathcal{O}_{\mathbb{P}^n}\arrow[r]&0
\end{tikzcd}\]



\end{proposition}
\end{tcolorbox}


\begin{tcolorbox}[colback=blue!5!white,colframe=blue!30!white]
\begin{proposition}
The \textbf{canonical bundle} of $\mathbb{P}^n$ is 
\[K_{\mathbb{P}^n}\cong \mathcal{O}_{\mathbb{P}^n}(-n-1)\]
\end{proposition}
\end{tcolorbox}


\section{Line Bundles and Divisors}
There is a bijection between vector bundles and locally free sheaves of rank $n$ on a complex manifold/variety. We outline the proof for the rank $1$ case:


\begin{tcolorbox}[colback=red!5!white,colframe=red!30!white]
\begin{theorem}
There is a bijection between 
\[\{\textrm{line bundles}\}\leftrightarrow \{\textrm{locally free sheaves of rank } 1\}\]




\end{theorem}
\end{tcolorbox}

\begin{proof}
    Given a line bundle $L\to X$, trivialized by $\{U_i\}$ and (holomorphic) transition functions $g_{ij}: U_i\cap U_j\to \mathbb{C}^*$, we construct its associated sheaf of sections: define the sheaf $F_L$ given by 
    \[F_L(U):= \{s: U\to L: s \textrm{ a local (holomorphic) section of } L\}\]
    Note that a local section on $U_i$ amounts to a (holomorphic) function 
    \[s: U_i\to \mathbb{C}\]
    so the restriction of $F_L$ to $U_i$ is isomorphic to the sheaf $(\mathcal{O}_X)_{U_i}$. Hence $F_L$ is the gluing of these locally free sheaves of rank $1$ via the transition functions $g_{ij}$: an isomorphism 
    \[\varphi_{ij}: (\mathcal{O}_X)|_{U_i\cap U_j}\to (\mathcal{O}_X)|_{U_j\cap U_i}\]
    given by 
    \[s_{i}\mapsto g_{ij}s_i\]
    and the cocycle condition is automatic. 

    Given a locally free sheaf $\mathcal{G}$ of rank $1$, find locally trivializations $U_i$ and isomorphisms 
    \[\mathcal{G}|_{U_i}\xrightarrow{\varphi_i} (\mathcal{O}_X)_{U_i}\] This gives rise to transition isomorphisms 
    \[\phi_{ij}:=\varphi_j\circ \varphi_i^{-1}: \varphi_{ij}: (\mathcal{O}_X)|_{U_i\cap U_j}\to (\mathcal{O}_X)|_{U_j\cap U_i}\]
    The isomorphism comes from an automorphism in $\mathcal{O}(U_{ij})^*$, which gives us a transition function to build a line bundle. 


\begin{tcolorbox}[colback=yellow!5!white,colframe=yellow!30!white]
\begin{example}
We want to understand the (holomorphic) line bundle/invertible sheaf $\mathcal{O}(-1)$ on $\mathbb{P}^n$. In term of line bundle, it is the subbundle 
\[O(-1):=\{(x,v)\in \mathbb{P}^n\times \mathbb{C}^{n+1}: x\in \mathbb{P}^n, v\in [x]\}\]
Its local trivialization is given by affine charts on $\mathbb{P}^n$: on $U_i$, a point $x\in U_i$ is expressed as 
\[x=[\frac{x_0}{x_i},\frac{x_1}{x_i},....,1,
\frac{x_{i+1}}{x_i},...,\frac{x_n}{x_i}]\]
and a vector on the line spanned by $x$ is of the form 
\[v=\lambda_i (\frac{x_0}{x_i},\frac{x_1}{x_i},....,1,
\frac{x_{i+1}}{x_i},...,\frac{x_n}{x_i})\]
and this gives us local trivialization. The transition function from $U_i$ to $U_j$ is then multiplication by $\frac{x_j}{x_i}$, since we have 
\[\lambda_i/x_i=\lambda_j/x_j\]
Locally $\frac{x_j}{x_i}$ is always holomorphic, so the line bundle is holomophic. Suppose 
\[s: \mathbb{P}^n\to \mathcal{O}(-1)\]
is a global section, then it restricts to local sections 
\[s_i: U_i=\mathbb{C}^n\to \mathbb{C} \]
and satisfies 
\[s_i=\frac{x_i}{x_j}s_j\]
But as we approch the hyperplane at infinity $x_i=0$, the section $s_i$ either has a pole, or it would be a constant by Liouville. But the right hand side is bounded when approching $x_i=0$ by holomophicity, therefore $s_i$ must be a constant. Doing this for other charts, we see that the local sections must all be constant, and the compatibility with transition functions says that they do not change when multiplied by $x_j/x_i$, which forces them to be $0$. 

\end{example}
\end{tcolorbox}


\begin{tcolorbox}[colback=yellow!5!white,colframe=yellow!30!white]
\begin{example}
We want to understand the dual bundle $\mathcal{O}(1)$ on $\mathbb{P}^n$. We may define it to be the dual bundle of $\mathcal{O}(-1)$: 
\[\mathcal{O}(1):= \textrm{Hom}(\mathcal{O}(-1), \mathcal{O}_X)\]
Geometrically, a fiber over $x\in \mathbb{P}^n$ is a linear functional on the line $[x]$. The transition function is the transpose of the inverse, which in our case, is $\frac{x_j}{x_i}$. If we repeat the analysis of local sections above 
\[s_i=\frac{x_j}{x_i}s_j\]
we see no obtructions as before. In fact, for any linear homogeneous function $f(x_0,...,x_n)$, we see that $s_i=f/x_i$ defines a local holomophic section on $U_i$, compatible with the transition functions. Conversely, if we are given a global holomorphic section $s$, we may lift this to a holomorphic function 
\[\tilde{s}: \mathbb{C}^{n+1}-\{0\}\to \mathbb{C}\]
defined by 
\[\tilde{s}(x_0,...,x_n)=x_is_i(x_0:...:x_n)\]
By Hartog's lemma, the function extends holomorphically over to $\mathbb{C}^{n+1}$, and is homogeneous of degree $1$, therefore must be a homogeneous polynoimal of degree $1$. 
\end{example}
\end{tcolorbox}

\end{proof}

\subsection{Divisors}


\begin{tcolorbox}[colback=purple!5!white,colframe=purple!75!black]
\begin{definition}
A \textbf{hypersurface} on a complex manifold $X$ is a subset $Y\subset X$ such that for every point $p\in Y$, there exists an open neighborhood $U\subset X$, and a regular function $f\in \mathcal{O}_X(U)$ such that 
\[Y\cap U= V(f)\]
\end{definition}
\end{tcolorbox}
In words, a hypersurface is a subset of $X$ locally cut out by a single holomorphic function. In particular, such hypersurface is not necessarily smooth. On the other hand, every codimension $1$ submanifold is locally cut out by the vanishing of a single local coordinate, so it is always a hypersurface. 







\begin{tcolorbox}[colback=purple!5!white,colframe=purple!75!black]
\begin{definition}
A \textbf{Weil divisor} is a formal $\mathbb{Z}$-linear combination of codimension $1$ irreducible hypersurfaces.
\end{definition}
\end{tcolorbox}
If $Y$ is a hypersurface with irreducible components $Y_i$, we can associate a Weil divisor to it 
\[[Y]=\sum [Y_i]\]

Given a irreducible hypersurface $Y$, we have the data of a covering $(U_i,f_i)$, where $f_i$ is a choice of irreducible local equation cutting out $Y$. This also gives us a compatibility condition on overlaps: if $U_i\cap U_j\cap Y$ is non-empty, then $f_i,f_j$ cut out the same hypersurface, they differ by a unit in $\mathcal{O}_X(U_i\cap U_j)$.(This is not so trivial to prove complex-analytically, maybe a DVR argument the stalk is easier). This is precisely the data of a Cartier divisor


\begin{tcolorbox}[colback=purple!5!white,colframe=purple!75!black]
\begin{definition}
A \textbf{Cartier divisor} on $X$ is the data of an open cover $U_i$ and a non-zero meromorphic function $f_i\in \mathcal{K}_X(U_i)$ such that on overlaps $U_i\cap U_j\neq \emptyset$, 
\[f_i= u_{ij} f_j\]
where $u_{ui}\in \mathcal{O}^*_X(U_i\cap U_j)$. Equivalently, this is an element in $H^0(X, \mathcal{K}_X^{*}/\mathcal{O}_X^{*})$. A Cartier divisor is \textbf{effective} if $f_i\in \mathcal{O}_X(U_i)$ for all $i$. 
\end{definition}
\end{tcolorbox}
In the definition, we see that we are also interested in the case where the local equation has poles instead of zeros. 


\begin{tcolorbox}[colback=blue!5!white,colframe=blue!30!white]
\begin{proposition}
There is a natural isomorphism 
\[\textrm{Cart}(X)\cong \textrm{Div}(X)\]
\end{proposition}
\end{tcolorbox}
The isomorphism is roughly as follows: given the data of a Cartier divisor $(U_i,f_i)$, each $f_i$ has a well-defined \textbf{order} on any hypersurface $Y$. Precisely, we define $\textrm{ord}_{Y,x}(f_i)$ as the integer such that
\[f_i=g^{\textrm{ord}_Y(f_i)}\cdot h\]
where $h\in \mathcal{O}^*_{X,x}$. It is the order of vanishing/pole of the function along $Y$. Conversely, given a Weil divisor $\sum n[Y_i]$, we know $Y_i$ is locally cut of by some irreducible $g$ on $U_i$ so this gives us the data $(U_i,\prod g_i^n)$ of a Cartier divisor. 


\section{Line bundle to Divisor}
Note that given the data of a Cartier divisor $D=(U_i, f_i)$, we see that 
\[g_{ij}=\frac{f_i}{f_j}\]
are all non-vanishing holomorphic functions on $U_i\cap U_j$, which gives rise to a transition function. Thus, we may define 

\begin{tcolorbox}[colback=purple!5!white,colframe=purple!75!black]
\begin{definition}
Given a Cartier divisor $D=(U_i,f_i)$ on $X$, its associate line bunble $\mathcal{O}_X(D)$ is given by the transition function \[g_{ij}=\frac{f_i}{f_j}\]
\end{definition}
\end{tcolorbox}


\begin{tcolorbox}[colback=yellow!5!white,colframe=yellow!30!white]
\begin{example}
Consider the hyperplane $H_0:=\{x_0=0\}$ on $\mathbb{P}^n$. Its associated Cartier divisor is $(U_i,f_i)$, where $U_i$ are the standard affines. The hyperplane does not intersect $H_0$, so we may take $f_0=1$. On other $U_i$, we see that $f_i=\frac{x_0}{x_i}$. Thus, the transition functions are 
\[g_{ij}= \frac{x_j}{x_i}\]
which we see is precisely the transition functions of $\mathcal{O}(1)$. Conversely, $H_0$ is cut out by the global section $x_0$ of $\mathcal{O}(1)$ glued from the local holomorphic sections $f_i$. If we think a little bit more regarding the example above, we see that there is nothing special with $H_0$: in fact taking any other hyperplane $H_i$ yield the same line bundle.  

\end{example}
\end{tcolorbox}

Let us describe the line bundle $\mathcal{O}_X(D)$ in terms of its local sections: on $U_i$, a local section is a holomorphic function $s_i\in \mathcal{O}_X(U_i)$, and on intersections we have 
\[\frac{f_i}{f_j}s_i=s_j\]

By taking $g_i=\frac{s_i}{f_i}$, we see this is equivalent to the data of meromorphic function $g$ such that $gf_i$ is holomorphic on each $U_i$, i.e a meromorphic function allowed to have poles controlled by $D$. 


\begin{tcolorbox}[colback=green!5!white,colframe=green!30!white]
\begin{remark}
This confuses me: if $D$ is effective, some global section of the line bundle $\mathcal{O}_X(D)$ should cut out $D$ when it is effective; but a general global section is allowed to have poles no worse than each $f_i$ on $D$. 
\end{remark}
\end{tcolorbox}

If $D$ is an effective divisor, there is a short exact sequnece of $\mathcal{O}_X$-modules,
\[0\to \mathcal{O}_X\xrightarrow{i} \mathcal{O}_X(D)\]
where on each $U_i$, $i$ is given by multiplication of the defining equation of $D$. The image of $i$ is the ideal sheaf $I_D$ defining $D$, so the cokernel is 
\[\mathcal{O}_D(D):= \mathcal{O}_X(D)/I_D \]
The notation suggests the sheaf $\mathcal{O}_D(D)$ on $X$ is supported on $D$, which is the case, and the pullback of $\mathcal{O}_D(D)$ along the inclusion $D\to X$ is the structure sheaf of $D$. The stalk of $\mathcal{O}_D(D)_x$ is then isomorphic to $\mathcal{O}_{X,x}/(f)$ if $x\in D$, and trivial otherwise. 



\subsection{Normal Bundle}
Given a codimension $1$ submanifold $i:D\in X$, we can define its normal bundle 
\[0\to TD\to i^*TX\to N_DX\to 0\]
as the cokernel. Its dual is called the \textbf{cornomal bundle}. Heuristically, a conormal vector is a form that vanishes on tangent vectors on $Y$. Locally, suppose we have a chart with coordinates $(z_0,,,,z_n)$ on which $D$ is cut out by the vanishing of $z_0$. A tangent vectors on $D$  linear combinations of $\{\frac{\partial}{\partial z_k} : k>0\}$. Thus, a one form vanishing on all these tangent vectors is of the form 
\[\alpha=g\cdot dz_0\]
More generally, the conormal sheaf should be the $\mathcal{O}_X$-module $I_D/(I_D)^2$, where $I_D$ is the ideal sheaf of functions vanishing on $D$. Locally, $I_D$ is generated by the defining function of $D$. The bijection of conormal vector and a function vanishing on $D$ is given by 
\[[f]\mapsto df\]
This is well-defined since $d(g^2)=2gdg$ vanishes identically on $D$. 

We then have the dual \textbf{cornomal exact sequence}
\[0\to I_D/(I_D)^2\to i^*(\Omega_X)\to \Omega_D\to 0\]

The line bundle $I_D/(I_D)^2$ has transition functions $\frac{f_j}{f_i}$, so it is dual to the bundle $\mathcal{O}_D(D)$. 




































\newpage
\bibliographystyle{plain}
\bibliography{Complex}

\end{document}