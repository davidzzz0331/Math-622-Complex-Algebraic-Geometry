\documentclass{article}
\usepackage[utf8]{inputenc}
\usepackage{amsmath}
\usepackage{amsfonts}
\usepackage{amssymb}
\usepackage{tikz}
\usepackage{fullpage}
\usepackage{tikz-cd}
\usepackage{spectralsequences}
\usepackage{adjustbox}
\usepackage{xfrac}
\usepackage{tcolorbox}
\usepackage{xcolor}
\usepackage{hyperref}
\usepackage{graphicx}
\graphicspath{ {D:/Chrome Downloads./} }
\usepackage[parfill]{parskip}
\usepackage{amsthm}
\usetikzlibrary{calc}
\theoremstyle{definition}
\newtheorem{theorem}{Theorem}[section]
\theoremstyle{definition}
\newtheorem{definition}{Definition}[theorem]
\theoremstyle{definition}
\newtheorem{remark}{Remark}[theorem]
\theoremstyle{definition}
\newtheorem{proposition}{Proposition}[theorem]
\theoremstyle{definition}
\newtheorem{lemma}[theorem]{Lemma}
\theoremstyle{definition}
\newtheorem{corollary}{Corollary}[theorem]
\theoremstyle{definition}
\newtheorem{example}{Example}[theorem]
\tikzset{curve/.style={settings={#1},to path={(\tikztostart)
    .. controls ($(\tikztostart)!\pv{pos}!(\tikztotarget)!\pv{height}!270:(\tikztotarget)$)
    and ($(\tikztostart)!1-\pv{pos}!(\tikztotarget)!\pv{height}!270:(\tikztotarget)$)
    .. (\tikztotarget)\tikztonodes}},
    settings/.code={\tikzset{quiver/.cd,#1}
        \def\pv##1{\pgfkeysvalueof{/tikz/quiver/##1}}},
    quiver/.cd,pos/.initial=0.35,height/.initial=0}
\title{Math 622: Complex Algebraic Geomtry}
\author{David Zhu}

\begin{document}
\maketitle

This is notes taken for Math 622: Complex Algebraic Geometry in Fall 2025, taught by Professor Ron Donagi at UPenn. 

\section{Overview of Hodge Theory}
\subsection{Wed 27}
The first part of the course will be a primer on complex algebraic geometry, in particular Hodge theory, without too much detailed proofs. 


\begin{tcolorbox}[colback=purple!5!white,colframe=purple!75!black]
\begin{definition}
A \textbf{Kahler manifold} is a complex maniolfd with a Hermitian metric whose imaginary part is closed. 
\end{definition}
\end{tcolorbox}


\begin{tcolorbox}[colback=yellow!5!white,colframe=yellow!30!white]
\begin{example}
The complex projective spaces $\mathbb{CP}^n$ with Fubini-Study metric are Kahler manifolds. Submanifolds of projective space inherit Kahler structures.
\end{example}
\end{tcolorbox}
A reference for this is \cite{Huy} 3.1.9.






\begin{tcolorbox}[colback=red!5!white,colframe=red!30!white]
\begin{theorem}[Kodaira Embedding]
A compact complex maniolf dadmits a holomorphic embedding into projective space iff it has a Kahler metric whose Kahler form is integral.
\end{theorem}
\end{tcolorbox}
The upshot is that Hodge and Lefshetz decomositions work for compact Kahler manifolds. 

Review of complex differential forms


\begin{tcolorbox}[colback=purple!5!white,colframe=purple!75!black]
\begin{definition}[Complex Differential Forms]

\end{definition}
\end{tcolorbox}





\begin{tcolorbox}[colback=purple!5!white,colframe=purple!75!black]
\begin{definition}
For a complex manifold $X$, we have a decomposition of differential forms
\[\Omega^k_{X,\mathbb{C}}:=\oplus_{p+q=k}\Omega^{p,q}_{X}\]
\end{definition}
\end{tcolorbox}



\begin{tcolorbox}[colback=red!5!white,colframe=red!30!white]
\begin{theorem}[Hodge Decomposition]
For a compact Kahler manifold
\[H^{k}(X,\mathbb{C})\cong \oplus_{p+q=k}H^{p,q}(X)\]
and 
\[H^{p,q}(X)\cong \overline{H^{p,q}(X)}\]
\end{theorem}
\end{tcolorbox}
The left hand side will be the DeRham cohomology, and the left hand side is the Dolbeaut cohomoology. 

\begin{tcolorbox}[colback=yellow!5!white,colframe=yellow!30!white]
\begin{example}[Hodge Diamond of a Riemann Surface]
    For a compact Riemann surafce $X$ of genus $g$, the Hodge numbers are 
    \[h^{0,0}=1, h^{1,0}=g, h^{0,1}=g, h^{1,1}=1\]


    \[\begin{tikzcd}
        & 1 \\
        g && g \\
        & 1
    \end{tikzcd}\]
\end{example}
\end{tcolorbox}


\begin{tcolorbox}[colback=yellow!5!white,colframe=yellow!30!white]
\begin{example}
For a compact complex surface $X$, the Hodge numbers are 
\[\begin{tikzcd}
	&& {h^{0,0}} \\
	& {h^{1,0}} & {} & {h^{0,1}} && {} \\
	{h^{2,0}} && {h^{1,1}} && {h^{0,2}} \\
	& {h^{2,1}} && {h^{1,2}} \\
	&& {h^{2,2}}
\end{tikzcd}\]
\end{example}
\end{tcolorbox}


\begin{tcolorbox}[colback=purple!5!white,colframe=purple!75!black]
\begin{definition}
Given the Kahler form $\omega$. define the Lefshetz operator 
\[L: H^{k}(X, \mathbb{R})\to H^{k+2}(X, \mathbb{R}) \]
given by 
\[\alpha\mapsto \omega \wedge \alpha\]
\end{definition}
\end{tcolorbox}


\begin{tcolorbox}[colback=red!5!white,colframe=red!30!white]
\begin{theorem}[Hard Lefshetz]
    For $n= dim X$, and $k\leq n$, we have 
    \[L^{n-k}: H^{k}(X, \mathbb{R})\cong H^{2n-k}(X, \mathbb{R})\]

\end{theorem}
\end{tcolorbox}


\begin{tcolorbox}[colback=purple!5!white,colframe=purple!75!black]
\begin{definition}[Primitive Cohomology]
    \[H^{k}(X, \mathbb{R})_{\textrm{prim}}:= ker(L^{n-k+1})\]
\end{definition}
\end{tcolorbox}


\subsection{Harmonic Forms}
\begin{enumerate}
    \item Equip $X$ with a Kahler metric. 
    \item Define the adjoint $d^*$ of $d$, and the Laplacian 
    \[\Delta_{d}=dd^*+d^*d\]
    \item The harmonic forms are 
    \[\mathcal{H}^k{X}:= ker \Delta_d\]
\end{enumerate}



\begin{tcolorbox}[colback=green!5!white,colframe=green!30!white]
\begin{corollary}[Harmonic Decomposition]
If $\alpha$ is harmonic, then each $(p,q)$-component of $\alpha$ is harmonic,. hnece 
\[\mathcal{H}(X)\cong \oplus_{p+q=k} \mathcal{H}^{p,q}(X)\]
\end{corollary}
\end{tcolorbox}

The Frolicher spectra seauence is obtained from the Hodge filtratiokn on the de Rham complex. 


\begin{tcolorbox}[colback=red!5!white,colframe=red!30!white]
\begin{theorem}
For a compact Kahler manifold, the Frolicher spectral sequence degenerates at $E_1$.
\end{theorem}
\end{tcolorbox}
The theorem implies the Hodge filtration on cohomology 
\[F^pH^k(X, \mathbb{C})= \oplus_{r\geq p}H^{r,k-r}(X)\]


\begin{tcolorbox}[colback=purple!5!white,colframe=purple!75!black]
\begin{definition}
A \textbf{Hodge structure} of \textbf{weight} $k$ is a decomposition 
\[H_{\mathbb{C}}=\oplus_{p+1=k}H^{p,q}\]
with $H^{p,q}=\overline{H^{p,q}}$
\end{definition}
\end{tcolorbox}

A story is that there is a dense open subset in $\mathbb{P}^9$ that parameterizes cubic curves in $\mathbb{P}^3$. Replacing the curves with their cohomology gives you a vector bundle over the base space. 



\begin{tcolorbox}[colback=purple!5!white,colframe=purple!75!black]
\begin{definition}
An \textbf{analytic cycle} of codimension $k$ is a linear combination of irreducible analytic subset of codimension $k$.
\end{definition}
\end{tcolorbox}

The Hodge conjecture states that if 
\[\alpha \in H^{2k}(X, \mathbb{Q})\cap H^{k,k}(X)\], then $\alpha$ is an algebraic cycle. 

\section{Wed, Sep 3}
Did not attend lol.
\section{Mon, Sep 8}
Review of complex manifolds and sheaves. 


\begin{tcolorbox}[colback=blue!5!white,colframe=blue!30!white]
\begin{proposition}
There is an equivalence between vector bundles and locally free $\mathcal{O}_X$-modules through sheaf of sections. 
\end{proposition}
\end{tcolorbox}


\begin{tcolorbox}[colback=purple!5!white,colframe=purple!75!black]
\begin{definition}
The \textbf{Picard group} of $X$ is the set of isomorphism classes of line bundles under tensor product. 
\[\textrm{Pic}(X)\cong H^1(X, \mathcal{O}^*_X)\]
\end{definition}
\end{tcolorbox}
The exponential sequence 
\[\begin{tikzcd}
0\arrow[r]&\mathbb{Z}\arrow[r]& \mathcal{O}_X\arrow[r]&\mathcal{O}^*_X\arrow[r]&0
\end{tikzcd}\]
induces the map 
\[c_1: \textrm{Pic}(X)\to H^2\]


\begin{tcolorbox}[colback=purple!5!white,colframe=purple!75!black]
\begin{definition}[\textbf{Canonical ring}]
We define
\[R(X)=\oplus_{m\geq 0} H^0(X,K_X^{\otimes m})\]
where $K_X$ is the canonical bundle. 
\end{definition}
\end{tcolorbox}


\begin{tcolorbox}[colback=purple!5!white,colframe=purple!75!black]
\begin{definition}
A sheaf $F$ is \textbf{flasque} is every restriction map 
\[F(X)\to F(U)\]
is surjective.
\end{definition}
\end{tcolorbox}


\begin{tcolorbox}[colback=green!5!white,colframe=green!30!white]
\begin{remark}
Quote from Ron: A flasque sheaf is almost never ``nice'': for example, the sheaf of continuous functions of a punctured disk is not flasque.
\end{remark}
\end{tcolorbox}


\begin{tcolorbox}[colback=blue!5!white,colframe=blue!30!white]
\begin{proposition}
Every sheaf has a flasque resolution. 
\end{proposition}
\end{tcolorbox}


\begin{tcolorbox}[colback=purple!5!white,colframe=purple!75!black]
\begin{definition}
The \textbf{sheaf cohomology} of a space with a sheaf $(M,F)$ is the cohomology of the chain complex obtained from takning global sections of any flasque resolution.  
\end{definition}
\end{tcolorbox}
As usual, it is independent of the chosen resolution, and a short exact sequence of sheaves induces a long exact sequence on cohomology. 


\begin{tcolorbox}[colback=purple!5!white,colframe=purple!75!black]
\begin{definition}
A sheaf is \textbf{acyclic} if all higher cohomologies are $0$. 
\end{definition}
\end{tcolorbox}


\begin{tcolorbox}[colback=red!5!white,colframe=red!30!white]
\begin{theorem}
For paracompact spaces, the Cech cohomology and sheaf cohomology are naturally isomorphic. 
\end{theorem}
\end{tcolorbox}


\begin{tcolorbox}[colback=blue!5!white,colframe=blue!30!white]
\begin{proposition}
For smooth line bundles, 
\[c_1: \textrm{Pic}(X)\to H^2(X; \mathbb{Z})\]
is an isomorphism, for $\mathcal{O}^*_X$ is \textbf{soft}, thus acyclic.
\end{proposition}
\end{tcolorbox}
For holomorphic case, this is \textbf{NOT} true. 


\begin{tcolorbox}[colback=blue!5!white,colframe=blue!30!white]
\begin{proposition}[Global Sections of $\mathcal{O}(k)$]
For $k>0$, there is an isomorphism 
\[H^0(\mathbb{P}^n, \mathcal{O}(k))\cong \mathbb{C}[z_0,...,z_n]_k\cong \textrm{Sym}^k*(V^{\vee})\]
\end{proposition}
\end{tcolorbox}



Let $T_{\mathbb{P}^n}$ and $\Omega^1_{\mathbb{P}^n}$ be the holomorphic tangent and cotangent bundles. 


\begin{tcolorbox}[colback=blue!5!white,colframe=blue!30!white]
\begin{proposition}[Euler Sequence]
On $\mathbb{P}^n$ there is a natural short exact sequence of holomorphic vector bundles 
\[\begin{tikzcd}
0\arrow[r]&\mathcal{O}_{\mathbb{P}^n}\arrow[r]& \mathcal{O}_{\mathbb{P}^n}(1)^{\oplus (n+1)}\arrow[r]& T_{\mathbb{P}^n}\arrow[r]&0
\end{tikzcd}\]

Alternatively, one may dualize and get 
\[ \begin{tikzcd}
0\arrow[r]&\Omega^1_{\mathbb{P}^n}\arrow[r]& \mathcal{O}_{\mathbb{P}^n}(-1)^{\oplus (n+1)}\arrow[r]&\mathcal{O}_{\mathbb{P}^n}\arrow[r]&0
\end{tikzcd}\]



\end{proposition}
\end{tcolorbox}


\begin{tcolorbox}[colback=blue!5!white,colframe=blue!30!white]
\begin{proposition}
The \textbf{canonical bundle} of $\mathbb{P}^n$ is 
\[K_{\mathbb{P}^n}\cong \mathcal{O}_{\mathbb{P}^n}(-n-1)\]
\end{proposition}
\end{tcolorbox}
















\newpage
\bibliographystyle{plain}
\bibliography{Complex}

\end{document}